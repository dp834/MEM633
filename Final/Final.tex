\documentclass{article}

\usepackage[margin=.75in]{geometry}
\usepackage{amsmath}
\usepackage{amssymb}
\usepackage[shortlabels]{enumitem}
\usepackage{float}
\usepackage{verbatim}
\usepackage{caption}
\usepackage{subcaption}
\usepackage{tikz}
\usetikzlibrary{shapes,arrows,positioning,calc}

\usepackage{mathtools}
\DeclarePairedDelimiter\norm{\lVert}{\rVert}%
\DeclarePairedDelimiter\abs{\lvert}{\rvert}%


\author{Damien Prieur}
\title{Final \\ MEM 633}
\date{}

\begin{document}
\maketitle

\section*{Problem 1}
Consider the following transfer function matrix
$$
H(s) = \frac{
    \begin{bmatrix}
        -s & s \\
        s+1 & 1 \\
    \end{bmatrix}
}{(s-1)(s+2)^2}
$$
\begin{enumerate}[(a)]
\item Find a state-space realization of $H(s)$ in block controller form.
\newline

\item Use the PBH Test to check the controllability and observability of the realization.
\newline

\item Is it a minimal realization?
If not, find a similarity transformation to transform the realization into either controllablility or observability decompostion.
Then find a minimal realization by eliminating the uncontrollable and/or unobservable parts of the system.
\newline

\item Determine the poles and zeros by using the $(A,B,C)$ matrices of the minimal realization in 1(c).
\newline
Note that the zeros of $(A,B,C)$ are the frequencies at which the rank of
$$
\begin{bmatrix}
sI-A & B \\
-C & D \\
\end{bmatrix}
$$
drops below its normal rank.
\newline

\end{enumerate}

\newpage
\section*{Problem 2}
For the transfer matrix $H(s)$ shown in Problem \#1, which can be represented as
$$ H(s) = N(s)D(s)^{-1} $$
Where
$$
N(s) =
\begin{bmatrix}
-s & s \\
s+1 & 1 \\
\end{bmatrix}
,\quad
D(s) =
(s-1)(s+2)^2
\begin{bmatrix}
1 & 0 \\
0 & 1 \\
\end{bmatrix}
$$
\begin{enumerate}[(a)]
\item Find a greatest common right divisor (gcrd) of $N(s)$ and $D(s)$.
\newline

\item Find an irreducible right MFD for $H(s)$ by extracting the gcrd of $N(s)$ and $D(s)$.
\newline

\item Determine the poles and zeros of the system based on the irreducible MFD in (b).
\newline

\item Find a state-space realization for the MFD in (c)
\newline

\item Is the state-space realization controllable and observable? Is it a minimal realization?
\newline

\item Can you find a similarity transformation which relates the realization in Problem \#1(c) and problem \#2(d)?
If yes, show the results and procedure. If not explain.
\newline

\end{enumerate}

\newpage
\section*{Problem 3}
Consider the following system,
\begin{figure}[!htb]
\centering
\tikzset{
    block/.style = {draw, fill=white, rectangle, minimum height=1em, minimum width=1em},
    tmp/.style  = {coordinate, node distance=1cm},
    sum/.style= {draw, fill=white, circle, node distance=1cm},
    input/.style = {coordinate, node distance=1cm},
    output/.style= {coordinate, node distance=2cm},
    pinstyle/.style = {pin edge={to-,thin,black}},
}
\begin{tikzpicture}[auto, node distance=2cm,>=latex']
\node [sum, name=inputSum ](inputSum){};
\node [block, right of=inputSum](controller){$K(s)$};
\node [block, right of=controller](transferFunc){$G(s)$};
\node [tmp, right of=transferFunc](tmpOutput) {};
\node [output, right of=tmpOutput](output) {};
\node [tmp, below of=transferFunc](belowTF) {};

\draw [->] (inputSum) --  (controller) {};
\draw [->] (controller) -- node[below]{$u(t)$} (transferFunc) {};
\draw [-] (transferFunc) -- (tmpOutput) {};
\draw [->] (tmpOutput) -- node[above]{$Y(s)$} node[below]{$y(t)$} (output) {};
\draw [-] (tmpOutput) |- (belowTF) {};
\draw [->] (belowTF) -| (inputSum) {};
\end{tikzpicture}
\end{figure}
where
$$ G(s) = \frac{s-2}{s^2-4s} $$
In the following you wil ldesign an observer-based controller $K(s)$ so that the closed-loop system is stable.
\begin{enumerate}[(a)]
\item Find a state space representation of $G(s)$.
\newline

\item Use the Riccati-equation approach to determine an observer based controller $K_1(s)$ such that the closed loop system is stable.
\newline

\item Find a state space representation of the closed loop system with $y(t)$ as the output.
\newline

\item Verify that the closed loop system poles are the regulator poles together with the observer poles.
\newline

\item Plot the state response for $x(t)$ and $u(t)$ due to the initial conditions $y(0)$, and $\dot{y}(0)$.
\newline

\item Repeat (b), (c), (d), and (e) with a different observer-based controller $K_2(s)$, which is obtained by choosing different weighting matrices in the Riccati equations.
\newline

\item Comment on how the weighting matrices in the Riccati equation and the pole location of the closed-loop system affect the closed loop system performance.
\newline

\end{enumerate}

\end{document}
