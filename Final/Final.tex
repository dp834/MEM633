\documentclass{article}

\usepackage[margin=.75in]{geometry}
\usepackage{amsmath}
\usepackage{amssymb}
\usepackage[shortlabels]{enumitem}
\usepackage{float}
\usepackage{verbatim}
\usepackage{caption}
\usepackage{subcaption}
\usepackage{tikz}
\usetikzlibrary{shapes,arrows,positioning,calc}

\usepackage{mathtools}
\DeclarePairedDelimiter\norm{\lVert}{\rVert}%
\DeclarePairedDelimiter\abs{\lvert}{\rvert}%


\author{Damien Prieur}
\title{Final \\ MEM 633}
\date{}

\begin{document}
\maketitle

\section*{Problem 1}
Consider the following transfer function matrix
$$
H(s) = \frac{
    \begin{bmatrix}
        -s & s \\
        s+1 & 1 \\
    \end{bmatrix}
}{(s-1)(s+2)^2}
$$
\begin{enumerate}[(a)]
\item Find a state-space realization of $H(s)$ in block controller form.
\newline
Using the block controller form we can just follow the procedure.
$$ y(s) = N(s) D_R(s)^{-1}u(s) $$
$$
N(s) =
\begin{bmatrix}
-s & s \\
s+1 & 1 \\
\end{bmatrix}
\quad
D_R(s) =
(s-1)(s+2)^2
\begin{bmatrix}
1 & 0 \\
0 & 1 \\
\end{bmatrix}
$$
Let $\xi(s) = D_R(s)^{-1}u(s)$ then we get
$$ y(s) = N(s) \xi(s) \qquad D_R(s)\xi(s) = u(s) $$
Expanding $N(s)$ and $d(s)$we get
$$
N(s) =
\begin{bmatrix}
-1 & 1 \\
1 & 0 \\
\end{bmatrix}
s
+
\begin{bmatrix}
0 & 0 \\
1 & 1 \\
\end{bmatrix}
\qquad
d(s) = s^3 + 3s^2 - 4
$$
Let $N_2(s) = 0$, $N_1(s)$ be the first term of $N(s)$ without the $s$ and $N_0(s)$ be the last term of $N(s)$
\newline
Taking the inverse laplace transform of $\xi(s)$ and $y(s)$ we get
$$
\dddot{\xi} =  - 3 \ddot{\xi} + 4\xi + u(t)
\qquad
y(t) =
\begin{bmatrix}
-1 & 1 \\
1 & 0 \\
\end{bmatrix}
\dot{\xi}
+
\begin{bmatrix}
0 & 0 \\
1 & 1 \\
\end{bmatrix}
\xi
$$
In matrix form we get
$$
\begin{bmatrix}
\dddot{\xi} \\
\ddot{\xi} \\
\dot{\xi} \\
\end{bmatrix}
=
\begin{bmatrix}
-3 & 0 & 4 \\
I & 0 & 0 \\
0 & I & 0 \\
\end{bmatrix}
\begin{bmatrix}
\ddot{\xi} \\
\dot{\xi} \\
\xi \\
\end{bmatrix}
+
\begin{bmatrix}
I \\
0 \\
0 \\
\end{bmatrix}
u(t)
$$
$$
y(t) =
\begin{bmatrix}
N_2 & N_1 & N_0
\end{bmatrix}
\begin{bmatrix}
\ddot{\xi} \\
\dot{\xi} \\
\xi \\
\end{bmatrix}
+
0 u(t)
$$
Now expanding $\xi = \begin{bmatrix} x_1 \\ x_2 \end{bmatrix}$ and plugging in we get
$$
\begin{bmatrix}
\dddot{x_1} \\
\dddot{x_2} \\
\ddot{x_1} \\
\ddot{x_2} \\
\dot{x_1} \\
\dot{x_2} \\
\end{bmatrix}
=
\begin{bmatrix}
-3 &  0 & 0 & 0 & 4 & 0 \\
 0 & -3 & 0 & 0 & 0 & 4 \\
 1 &  0 & 0 & 0 & 0 & 0 \\
 0 &  1 & 0 & 0 & 0 & 0 \\
 0 &  0 & 1 & 0 & 0 & 0 \\
 0 &  0 & 0 & 1 & 0 & 0 \\
\end{bmatrix}
\begin{bmatrix}
\ddot{x_1} \\
\ddot{x_2} \\
\dot{x_1} \\
\dot{x_2} \\
x_1 \\
x_2 \\
\end{bmatrix}
+
\begin{bmatrix}
1 & 0  \\
0 & 1  \\
0 & 0  \\
0 & 0  \\
0 & 0  \\
0 & 0  \\
\end{bmatrix}
u(t)
$$
$$
y(t) =
\begin{bmatrix}
0 & 0 & -1 & 1 & 0 & 0 \\
0 & 0 &  1 & 0 & 1 & 1 \\
\end{bmatrix}
\begin{bmatrix}
\ddot{x_1} \\
\ddot{x_2} \\
\dot{x_1} \\
\dot{x_2} \\
x_1 \\
x_2 \\
\end{bmatrix}
$$

\item Use the PBH Test to check the controllability and observability of the realization.
\newline
By the PBH test we know that $\{A,B\}$ is controllable if and only if $rank[sI-A \quad B] = n \quad \forall s$ where $n$ is the size of $A$
$$
\begin{bmatrix}
s+3 &   0 & 0 & 0 &-4 & 0 & 1 & 0\\
  0 & s+3 & 0 & 0 & 0 &-4 & 0 & 1\\
 -1 &   0 & s & 0 & 0 & 0 & 0 & 0\\
  0 &  -1 & 0 & s & 0 & 0 & 0 & 0\\
  0 &   0 &-1 & 0 & s & 0 & 0 & 0\\
  0 &   0 & 0 &-1 & 0 & s & 0 & 0\\
\end{bmatrix}
$$
We know that if $s$ is not an eigenvalue then the $A$ matrix is already going to be full rank, so we only need to check the eigenvalues of $A$
$$eign(A) = \{ -2, 1 \}$$
These eigenvalues are have multiplicity 4 and 2 respectively.
Plugging in these values we get
$$
s \to -2 =
\begin{bmatrix}
  1 &   0 & 0 & 0 &-4 & 0 & 1 & 0\\
  0 &   1 & 0 & 0 & 0 &-4 & 0 & 1\\
 -1 &   0 &-2 & 0 & 0 & 0 & 0 & 0\\
  0 &  -1 & 0 &-2 & 0 & 0 & 0 & 0\\
  0 &   0 &-1 & 0 &-2 & 0 & 0 & 0\\
  0 &   0 & 0 &-1 & 0 &-2 & 0 & 0\\
\end{bmatrix}
\quad
s \to 1 =
\begin{bmatrix}
  4 &   0 & 0 & 0 &-4 & 0 & 1 & 0\\
  0 &   4 & 0 & 0 & 0 &-4 & 0 & 1\\
 -1 &   0 & 1 & 0 & 0 & 0 & 0 & 0\\
  0 &  -1 & 0 & 1 & 0 & 0 & 0 & 0\\
  0 &   0 &-1 & 0 & 1 & 0 & 0 & 0\\
  0 &   0 & 0 &-1 & 0 & 1 & 0 & 0\\
\end{bmatrix}
$$
Which both have rank$= 6$ which is full rank.
\fbox{So the system is controllable.}
\newline
Now for observability from the PBH test we know that a system is controllable if and only if $rank\begin{bmatrix} sI-A \\ C \end{bmatrix} = n \quad \forall s$ where $n$ is the size of $A$
$$
\begin{bmatrix}
s+3 &   0 & 0 & 0 &-4 & 0 \\
  0 & s+3 & 0 & 0 & 0 &-4 \\
 -1 &   0 & s & 0 & 0 & 0 \\
  0 &  -1 & 0 & s & 0 & 0 \\
  0 &   0 &-1 & 0 & s & 0 \\
  0 &   0 & 0 &-1 & 0 & s \\
  0 &   0 &-1 & 1 & 0 & 0 \\
  0 &   0 & 1 & 0 & 1 & 1 \\
\end{bmatrix}
$$
Again we only need to look at the eigenvalues.
$$
s \to -2 =
\begin{bmatrix}
  1 &   0 & 0 & 0 &-4 & 0 \\
  0 &   1 & 0 & 0 & 0 &-4 \\
 -1 &   0 &-2 & 0 & 0 & 0 \\
  0 &  -1 & 0 &-2 & 0 & 0 \\
  0 &   0 &-1 & 0 &-2 & 0 \\
  0 &   0 & 0 &-1 & 0 &-2 \\
  0 &   0 &-1 & 1 & 0 & 0 \\
  0 &   0 & 1 & 0 & 1 & 1 \\
\end{bmatrix}
\quad
s \to 1 =
\begin{bmatrix}
  4 &   0 & 0 & 0 &-4 & 0 \\
  0 &   4 & 0 & 0 & 0 &-4 \\
 -1 &   0 & 1 & 0 & 0 & 0 \\
  0 &  -1 & 0 & 1 & 0 & 0 \\
  0 &   0 &-1 & 0 & 1 & 0 \\
  0 &   0 & 0 &-1 & 0 & 1 \\
  0 &   0 &-1 & 1 & 0 & 0 \\
  0 &   0 & 1 & 0 & 1 & 1 \\
\end{bmatrix}
$$
The first eigenvalue $-2$ only has rank $5$ while the second has rank $6$. \fbox{The system is not fully observable.}


\item Is it a minimal realization?
If not, find a similarity transformation to transform the realization into either controllablility or observability decompostion.
Then find a minimal realization by eliminating the uncontrollable and/or unobservable parts of the system.
\newline
Since it's not both controllable and observable it's not a minimal realization.
Since we know that the observability is the issue we can perform observability decomposition to remove the unobservable parts of the system.
Following the process for observability decomposition we choose $T^{-1} = \begin{bmatrix}U_1 \\ U_2 \end{bmatrix}$ such that $U_1$ are $5$ linearly independent row vectors of the observability matrix and $U_2$ makes square matrix full rank.
If we take the first 5 rows of the observability matrix they are linearly independent.
$$
T^{-1} =
\begin{bmatrix}
-1 & 1 & 0 & 0 & 0 & 0 \\
1 & 0 & 1 & 1 & 0 & 0 \\
3 & -3 & 0 & 0 & -4 & 4 \\
-2 & 1 & 0 & 0 & 4 & 0 \\
-9 & 9 & -4 & 4 & 12 & -12 \\
1 & 0 & 0 & 0 & 0 & 0 \\
\end{bmatrix}
\implies
T =
\begin{bmatrix}
0 & 0 & 0 & 0 & 0 & 1 \\
1 & 0 & 0 & 0 & 0 & 1 \\
0 & \frac{1}{2} & -\frac{3}{8} & 0 & -\frac{1}{8} & -\frac{1}{2} \\
0 & \frac{1}{2} & \frac{3}{8} & 0 & \frac{1}{8} & -\frac{1}{2} \\
-\frac{1}{4} & 0 & 0 & \frac{1}{4} & 0 & \frac{1}{4} \\
\frac{1}{2} & 0 & \frac{1}{4} & \frac{1}{4} & 0 & \frac{1}{4} \\
\end{bmatrix}
$$
Applying the simlarity transform $T$ as $\{\bar{A} = T^{-1}AT, \bar{B}=T^{-1}B, \bar{C}=CT\}$
$$
\bar{A} =
\begin{bmatrix}
0 & 0 & 1 & 0 & 0 & 0 \\
0 & 0 & 0 & 1 & 0 & 0 \\
0 & 0 & 0 & 0 & 1 & 0 \\
1 & 2 & -\frac{1}{2} & -1 & -\frac{1}{2} & 0 \\
4 & 0 & 0 & 0 & -3 & 0 \\
-1 & 0 & 0 & 1 & 0 & -2 \\
\end{bmatrix}
\quad
\bar{B} =
\begin{bmatrix}
-1 & 1 \\
 1 & 0 \\
 3 & -3 \\
-2 & 1 \\
-9 & 9 \\
 1 & 0 \\
\end{bmatrix}
\quad
\bar{C} =
\begin{bmatrix}
0 & 0 & \frac{3}{4} & 0 & \frac{1}{4} & 0 \\
\frac{1}{4} & \frac{1}{2} & -\frac{1}{8} & \frac{1}{2} & -\frac{1}{8} & 0 \\
\end{bmatrix}
$$
If we take the $5x5$ subsystem and their corresponding terms in the $\bar{B}$ and $\bar{C}$ matricies we get an observable subsystem.
$$
\bar{A}_{min} =
\begin{bmatrix}
0 & 0 & 1 & 0 & 0 \\
0 & 0 & 0 & 1 & 0 \\
0 & 0 & 0 & 0 & 1 \\
1 & 2 & -\frac{1}{2} & -1 & -\frac{1}{2} \\
4 & 0 & 0 & 0 & -3 \\
\end{bmatrix}
\quad
\bar{B}_{min} =
\begin{bmatrix}
-1 & 1 \\
 1 & 0 \\
 3 & -3 \\
-2 & 1 \\
-9 & 9 \\
\end{bmatrix}
\quad
\bar{C}_{min} =
\begin{bmatrix}
0 & 0 & \frac{3}{4} & 0 & \frac{1}{4} \\
\frac{1}{4} & \frac{1}{2} & -\frac{1}{8} & \frac{1}{2} & -\frac{1}{8} \\
\end{bmatrix}
$$
If we compute $\bar{C}_{min}\left(s I - \bar{A}_{min} \right)^{-1}\bar{B}_{min}$ we can verify that it gives the same transfer function we started with.

\item Determine the poles and zeros by using the $(A,B,C)$ matrices of the minimal realization in 1(c).
\newline
Note that the zeros of $(A,B,C)$ are the frequencies at which the rank of
$$
\begin{bmatrix}
sI-A & B \\
-C & D \\
\end{bmatrix}
$$
drops below its normal rank.
\newline
\newline
We can find the poles as the eigenvalues of the $A$ matrix.
$$
eig\left(
\begin{bmatrix}
0 & 0 & 1 & 0 & 0 \\
0 & 0 & 0 & 1 & 0 \\
0 & 0 & 0 & 0 & 1 \\
1 & 2 & -\frac{1}{2} & -1 & -\frac{1}{2} \\
4 & 0 & 0 & 0 & -3 \\
\end{bmatrix}
\right)
=
\{-2, -2, -2, 1, 1\}
$$
For the poles we look at
$$
\begin{bmatrix}
sI-A & B \\
-C & D \\
\end{bmatrix}
=
\begin{bmatrix}
s & 0 & -1 & 0 & 0 & -1 & 1 \\
0 & s & 0 & -1 & 0 & 1 & 0 \\
0 & 0 & s & 0 & -1 & 3 & -3 \\
-1 & -2 & \frac{1}{2} & s+1 & \frac{1}{2} & -2 & 1 \\
-4 & 0 & 0 & 0 & s+3 & -9 & 9 \\
0 & 0 & -\frac{3}{4} & 0 & -\frac{1}{4} & 0 & 0 \\
-\frac{1}{4} & -\frac{1}{2} & \frac{1}{8} & -\frac{1}{2} & \frac{1}{8} & 0 & 0 \\
\end{bmatrix}
$$
We only need to look at when the determinant is zero to find where the matrix becomes rank deficient.
$$
det\left(
\begin{bmatrix}
s & 0 & -1 & 0 & 0 & -1 & 1 \\
0 & s & 0 & -1 & 0 & 1 & 0 \\
0 & 0 & s & 0 & -1 & 3 & -3 \\
-1 & -2 & \frac{1}{2} & s+1 & \frac{1}{2} & -2 & 1 \\
-4 & 0 & 0 & 0 & s+3 & -9 & 9 \\
0 & 0 & -\frac{3}{4} & 0 & -\frac{1}{4} & 0 & 0 \\
-\frac{1}{4} & -\frac{1}{2} & \frac{1}{8} & -\frac{1}{2} & \frac{1}{8} & 0 & 0 \\
\end{bmatrix}
\right)
=
-s
$$
So our only zero for the system is when $-s = 0$ or when $s=0$.
\newline
\fbox{The zeros of our system are $s = 0$ and the poles are $s = \{-2, -2, -2, 1, 1\}$}

\end{enumerate}

\newpage
\section*{Problem 2}
For the transfer matrix $H(s)$ shown in Problem \#1, which can be represented as
$$ H(s) = N(s)D(s)^{-1} $$
Where
$$
N(s) =
\begin{bmatrix}
-s & s \\
s+1 & 1 \\
\end{bmatrix}
,\quad
D(s) =
(s-1)(s+2)^2
\begin{bmatrix}
1 & 0 \\
0 & 1 \\
\end{bmatrix}
$$
\begin{enumerate}[(a)]
\item Find a greatest common right divisor (gcrd) of $N(s)$ and $D(s)$.
\newline

\item Find an irreducible right MFD for $H(s)$ by extracting the gcrd of $N(s)$ and $D(s)$.
\newline

\item Determine the poles and zeros of the system based on the irreducible MFD in (b).
\newline

\item Find a state-space realization for the MFD in (c)
\newline

\item Is the state-space realization controllable and observable? Is it a minimal realization?
\newline

\item Can you find a similarity transformation which relates the realization in Problem \#1(c) and problem \#2(d)?
If yes, show the results and procedure. If not explain.
\newline

\end{enumerate}

\newpage
\section*{Problem 3}
Consider the following system,
\begin{figure}[!htb]
\centering
\tikzset{
    block/.style = {draw, fill=white, rectangle, minimum height=1em, minimum width=1em},
    tmp/.style  = {coordinate, node distance=1cm},
    sum/.style= {draw, fill=white, circle, node distance=1cm},
    input/.style = {coordinate, node distance=1cm},
    output/.style= {coordinate, node distance=2cm},
    pinstyle/.style = {pin edge={to-,thin,black}},
}
\begin{tikzpicture}[auto, node distance=2cm,>=latex']
\node [sum, name=inputSum ](inputSum){};
\node [block, right of=inputSum](controller){$K(s)$};
\node [block, right of=controller](transferFunc){$G(s)$};
\node [tmp, right of=transferFunc](tmpOutput) {};
\node [output, right of=tmpOutput](output) {};
\node [tmp, below of=transferFunc](belowTF) {};

\draw [->] (inputSum) --  (controller) {};
\draw [->] (controller) -- node[below]{$u(t)$} (transferFunc) {};
\draw [-] (transferFunc) -- (tmpOutput) {};
\draw [->] (tmpOutput) -- node[above]{$Y(s)$} node[below]{$y(t)$} (output) {};
\draw [-] (tmpOutput) |- (belowTF) {};
\draw [->] (belowTF) -| (inputSum) {};
\end{tikzpicture}
\end{figure}
where
$$ G(s) = \frac{s-2}{s^2-4s} $$
In the following you wil ldesign an observer-based controller $K(s)$ so that the closed-loop system is stable.
\begin{enumerate}[(a)]
\item Find a state space representation of $G(s)$.
\newline

\item Use the Riccati-equation approach to determine an observer based controller $K_1(s)$ such that the closed loop system is stable.
\newline

\item Find a state space representation of the closed loop system with $y(t)$ as the output.
\newline

\item Verify that the closed loop system poles are the regulator poles together with the observer poles.
\newline

\item Plot the state response for $x(t)$ and $u(t)$ due to the initial conditions $y(0)$, and $\dot{y}(0)$.
\newline

\item Repeat (b), (c), (d), and (e) with a different observer-based controller $K_2(s)$, which is obtained by choosing different weighting matrices in the Riccati equations.
\newline

\item Comment on how the weighting matrices in the Riccati equation and the pole location of the closed-loop system affect the closed loop system performance.
\newline

\end{enumerate}

\end{document}
