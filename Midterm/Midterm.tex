\documentclass{article}

\usepackage[margin=.75in]{geometry}
\usepackage{amsmath}
\usepackage{amssymb}
\usepackage[shortlabels]{enumitem}
\usepackage{float}
\usepackage{verbatim}
\usepackage{caption}
\usepackage{subcaption}
\usepackage{tikz}
\usetikzlibrary{shapes,arrows,positioning,calc}

\usepackage{mathtools}
\DeclarePairedDelimiter\norm{\lVert}{\rVert}%

\author{Damien Prieur}
\title{Midterm \\ MEM 633}
\date{}

\begin{document}
\maketitle

\section*{Problem 1}
For the nonlinear cart-inverted pendulum system shown in page 11 of Chapter 1 notes, a linearized model at the upright stick equilibrium can be obtained as follows if both of the rotational and translational friction coefficients are negligible.
$$
\dot{x}(t) =
\begin{bmatrix}
0 & 1 & 0 & 0 \\
25 & 0 & 0 & 0 \\
0 & 0 & 0 & 1 \\
-0.4 & 0 & 0 & 0 \\
\end{bmatrix}
x(t) +
\begin{bmatrix}
0 \\
-1.4 \\
0 \\
0.6 \\
\end{bmatrix}
u(t)
$$
Assume the initial condition is
$$
x(0) =
\begin{bmatrix}
-0.3 \text{rad} \\
0 \\
0.2m \\
0 \\
\end{bmatrix}
$$

\begin{enumerate}[(a)]
\item Is the system stable? Controllable? Explain.
\newline
\newline

\item Let $u(t) = 0$. Solve for $x(t)$ and then plot it and comment on the initial state response.
\newline
\newline

\item Now, use the state feedback $u(t) = Fx(t)$ with $F = \begin{bmatrix} 139 & 24 & 30 & 31 \end{bmatrix}$ to stabilize the system.
Find the closed-loop pole locations of the closed-loop system to verify that all these poles are in the left half of the complex plane.
\newline
\newline

\item Plot the displacement of the state variables $x_1(t)$ and $x_3(t)$ on the first graph, the velocity state variables $x_2(t)$ and $x_4(t)$ on the second graph, and the control input $u(t)$ on the third graph for the closed-loop system.
\newline
\newline

\item Comment on the physical meaning of the three time response graphs in (d).
\newline
\newline

\end{enumerate}

\newpage
\section*{Problem 2}
Consider the following block diagram
\begin{figure}[!htb]
\centering
\tikzset{
    block/.style = {draw, fill=white, rectangle, minimum height=1em, minimum width=1em},
    tmp/.style  = {coordinate, node distance=1cm},
    sum/.style= {draw, fill=white, circle, node distance=1cm},
    input/.style = {coordinate, node distance=1cm},
    output/.style= {coordinate, node distance=2cm},
    pinstyle/.style = {pin edge={to-,thin,black}},
}
\begin{tikzpicture}[auto, node distance=2cm,>=latex']
\node [input, name=input](input){};
\node [block, right of=input](transferFunc){$G(s)$};
\node [output, right of=transferFunc](output) {};

\draw [->] (input) -- node[above]{$U(s)$} node[below]{$u(t)$} (transferFunc) {};
\draw [->] (transferFunc) -- node[above]{$Y(s)$} node[below]{$y(t)$} (output) {};
\end{tikzpicture}
\end{figure}
where $G(s)$ is given as
$$ G(s) = \frac{s-2}{s(s^2+1)} $$
\begin{enumerate}[(a)]
\item Explain why the system is not BIBO stable based on the \textbf{definition} of BIBO stability, and find two bounded inputs that would cause the output to be unbounded.
Plot the output due to these two inputs.
\newline
\newline

\item Find a state-space representation in the controller canonical for for the system $G(s)$.
\begin{align*}
\dot{x}(t) &= Ax(t) + Bu(t) \\
y(t) &= Cx(t) + Du(t)
\end{align*}
\newline

\item Explain why the system is not internally stable based on the \textbf{definition} of internal stability.
Plot the state response, $x(t)$, of the system with initial state $x(0) = \begin{bmatrix}2 & 0 & -2 \end{bmatrix}^T$ and zero input.
\newline
\newline

\end{enumerate}

\newpage
\section*{Problem 3}
Consider the following feedback control system.
\begin{figure}[!htb]
\centering
\tikzset{
    block/.style = {draw, fill=white, rectangle, minimum height=1em, minimum width=1em},
    tmp/.style  = {coordinate, node distance=1cm},
    sum/.style= {draw, fill=white, circle, node distance=1cm},
    input/.style = {coordinate, node distance=1cm},
    output/.style= {coordinate, node distance=2cm},
    pinstyle/.style = {pin edge={to-,thin,black}},
}
\begin{tikzpicture}[auto, node distance=2cm,>=latex']
\node [input, name=input](input){};
\node [sum, right of=input, label=south east:{$-$}](inputSum){};
\node [block, right of=inputSum](controller){$K(s)$};
\node [block, right of=controller](transferFunc){$G(s)$};
\node [tmp, right of=transferFunc](tmpOutput) {};
\node [output, right of=tmpOutput](output) {};
\node [tmp, below of=transferFunc](belowTF) {};

\draw [->] (input) -- node[above]{$R(s)$} node[below]{$r(t)$} (inputSum) {};
\draw [->] (inputSum) --  (controller) {};
\draw [->] (controller) -- (transferFunc) {};
\draw [-] (transferFunc) -- (tmpOutput) {};
\draw [->] (tmpOutput) -- node[above]{$Y(s)$} node[below]{$y(t)$} (output) {};
\draw [-] (tmpOutput) |- (belowTF) {};
\draw [->] (belowTF) -| (inputSum) {};
\end{tikzpicture}
\end{figure}
Given
$$G(s) = \frac{s-1}{s^2-2s}$$

\begin{enumerate}[(a)]
\item Find a controller $K(s)$ with the least order, which is either strictly proper or proper, so that the closed-loop system is BIBO stable.
Hint: You can start from a zero-order controller $K_0(s) = b_0$. If it does not work, try a $1^{\text{st}}$-order controller $ K_1(s) = \frac{b_1 s + b_0}{s+a_0}$ or a $2^{\text{nd}}$-order controller $K_2(s) = \frac{b_2 s^2 + b_1 s + b_0}{s^2 + a_1 s + a_0}$.
\newline
\newline

\item With the controller $K(s)$ you have just designed, determine the closed-loop transfer function $\frac{Y(s)}{R(s)}$.
\newline
\newline

\item Assume zero initial conditions and the reference input $r(t)$ is a unit step function.
Plot the output response $y(t)$ of the feedback control system you have just designed.
\newline
\newline

\end{enumerate}

\newpage
\section*{Problem 4}
Consider the system described by the following state equation,
$$ \dot{x}(t) =
\begin{bmatrix}
-21 & -22 & -20 \\
26 & 27 & 23 \\
-9 & -9 & -7 \\
\end{bmatrix}
x(t) +
\begin{bmatrix}
-3 \\
5 \\
-2 \\
\end{bmatrix}
u(t)
$$

\begin{enumerate}[(a)]
\item Use PBH Test to check if the system is controllable.
\newline
\newline

\item Characterize the controllable subspace.
\newline
\newline

\item If possible, design a state feedback controller so that the closed-loop system is internally stable.
If not, explain the reason.
\newline
\newline

\end{enumerate}

\end{document}
