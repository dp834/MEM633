\documentclass{article}

\usepackage[margin=.75in]{geometry}
\usepackage{amsmath}
\usepackage{amssymb}
\usepackage[shortlabels]{enumitem}
\usepackage{float}
\usepackage{verbatim}
\usepackage{caption}
\usepackage{subcaption}
\usepackage{tikz}
\usetikzlibrary{shapes,arrows,positioning,calc}

\usepackage{mathtools}
\DeclarePairedDelimiter\norm{\lVert}{\rVert}%
\DeclarePairedDelimiter\abs{\lvert}{\rvert}%


\author{Damien Prieur}
\title{Midterm \\ MEM 633}
\date{}

\begin{document}
\maketitle

\section*{Problem 1}
For the nonlinear cart-inverted pendulum system shown in page 11 of Chapter 1 notes, a linearized model at the upright stick equilibrium can be obtained as follows if both of the rotational and translational friction coefficients are negligible.
$$
\dot{x}(t) =
\begin{bmatrix}
0 & 1 & 0 & 0 \\
25 & 0 & 0 & 0 \\
0 & 0 & 0 & 1 \\
-0.4 & 0 & 0 & 0 \\
\end{bmatrix}
x(t) +
\begin{bmatrix}
0 \\
-1.4 \\
0 \\
0.6 \\
\end{bmatrix}
u(t)
$$
Assume the initial condition is
$$
x(0) =
\begin{bmatrix}
-0.3 \text{rad} \\
0 \\
0.2m \\
0 \\
\end{bmatrix}
$$

\begin{enumerate}[(a)]
\item Is the system stable? Controllable? Explain.
\newline
\newline
A stable system has all eigenvalues in the left-half plane as it causes all the terms to exponentially decay due to the negative real part.
Finding the characteristic polynomial by expanding the determinant along the fourth column gives us
$$
det(A-\lambda I) =
\begin{vmatrix}
-\lambda & 1 & 0 & 0 \\
25 & -\lambda & 0 & 0 \\
0 & 0 & -\lambda & 1 \\
-0.4 & 0 & 0 & -\lambda \\
\end{vmatrix}
=
\lambda^4 - 25\lambda^2
=0
$$
Solving this give us
$$ \lambda^4 = 25\lambda^2 $$
$$ \lambda^2 = 25 $$
$$ \lambda = \pm 5 $$
One trivial solution is $0$ which is repeated twice and $\pm 5$.
Since we have eigenvalues that are $\ge 0$ for the real component the system is unstable as we expect.
\newline
To determine controllability we can look at the controllability matrix given by
$$ \mathcal{C} = \begin{bmatrix} B & AB & A^2B & A^3B \end{bmatrix}$$
If the matrix is full rank the system is controllable since the input can have decoupled control over all state space variables.
Using Mathematica to compute these matrix multiplications we get
$$ \mathcal{C} =
\begin{bmatrix}
   0 & -1.4 & 0 & -35 \\
-1.4 & 0 & -35 & 0 \\
   0 & 0.6 & 0 & 0.56 \\
 0.6 & 0 & 0.56 & 0 \\
\end{bmatrix}
$$
A matrix is full rank if the determinant is nonzero, so computing the determinant we get
$$ rank(\mathcal{C}) =
\begin{vmatrix}
   0 & -1.4 & 0 & -35 \\
-1.4 & 0 & -35 & 0 \\
   0 & 0.6 & 0 & 0.56 \\
 0.6 & 0 & 0.56 & 0 \\
\end{vmatrix}
=
408.687
$$
Since it's nonzero the controllability matrix is full rank. Therefore the system is controllable.

\item Let $u(t) = 0$. Solve for $x(t)$ and then plot it and comment on the initial state response.
\newline
\newline
Since the system is a linear system the solution to $x(t)$ is
$$ x(t) = e^{A t} x(t_0) + \int_0^t \Phi(t,\tau)B(\tau)u(\tau)d\tau $$
Since the input is $0$ we just need the first term
$$ x(t) = e^{A t} x(t_0) $$
So first we must compute $e^{A t}$.
Since we have a repeated eigenvalue $0$ the best we can do for digitalization is Jordan decomposition.
We have the jordan blocks as
$$
J=
\begin{bmatrix}
-5& 0 & 0 & 0 \\
0 & 0 & 1 & 0 \\
0 & 0 & 0 & 0 \\
0 & 0 & 0 & 5 \\
\end{bmatrix}
$$
The corresponding generalized eigenvectors computed with Mathematica give
$$
S
\begin{bmatrix}
0.196 & 0 & 0 & -0.196 \\
-0.980 & 0 & 0 & -0.980 \\
-0.00314 & 1 & 0 & 0.00314 \\
0.0157 & 0 & 1 & 0.0157 \\
\end{bmatrix}
$$
Since we now have $A = SJS^{-1}$ we can find the matrix exponential by computing $e^{A t} = Se^{J t}S^{-1}$ which is easy to compute since $J$ is block diagonal.
$$
e^{J t} =
\begin{bmatrix}
e^{-5 t} & 0 & 0 & 0 \\
0 & e^{0t} & te^{0} & 0 \\
0 & 0 & e^{0 t} & 0 \\
0 & 0 & 0 & e^{5 t} \\
\end{bmatrix}
=
\begin{bmatrix}
e^{-5 t} & 0 & 0 & 0 \\
0 & 1 & t & 0 \\
0 & 0 & 1 & 0 \\
0 & 0 & 0 & e^{5 t} \\
\end{bmatrix}
$$
Now computing $e^{A t} = S e^{J t} S^{-1}$ in Mathematica we get
$$
e^{A t} =
\begin{bmatrix}
0.5 e^{-5. t}+0.5 e^{5. t} & 0.1 e^{5. t}-0.1 e^{-5. t} & 0 & 0 \\
2.5 e^{5. t}-2.5 e^{-5. t} & 0.5 e^{-5. t}+0.5 e^{5. t} & 0 & 0 \\
0.016\, -0.008 e^{-5. t}-0.008 e^{5. t} & 0.016 t+0.0016 e^{-5. t}-0.0016 e^{5. t} & 1 & t \\
0.04 e^{-5. t}-0.04 e^{5. t} & 0.016\, -0.008 e^{-5. t}-0.008 e^{5. t} & 0 & 1 \\
\end{bmatrix}
$$
Now to solve for $x(t)$ we have
$$
x(t) =
\begin{bmatrix}
0.5 e^{-5. t}+0.5 e^{5. t} & 0.1 e^{5. t}-0.1 e^{-5. t} & 0 & 0 \\
2.5 e^{5. t}-2.5 e^{-5. t} & 0.5 e^{-5. t}+0.5 e^{5. t} & 0 & 0 \\
0.016\, -0.008 e^{-5. t}-0.008 e^{5. t} & 0.016 t+0.0016 e^{-5. t}-0.0016 e^{5. t} & 1 & t \\
0.04 e^{-5. t}-0.04 e^{5. t} & 0.016\, -0.008 e^{-5. t}-0.008 e^{5. t} & 0 & 1 \\
\end{bmatrix}
\begin{bmatrix}
-0.3 \text{rad} \\
0 \\
0.2m \\
0 \\
\end{bmatrix}
$$
Which gives us
$$
\begin{bmatrix}
x_1(t) \\
x_2(t) \\
x_3(t) \\
x_4(t) \\
\end{bmatrix}
=
\begin{bmatrix}
e^{-5. t} \left(-0.15 e^{10. t}-0.15\right) \\
e^{-5. t} \left(0.75\, -0.75 e^{10. t}\right) \\
0.1952 + 0.0024 e^{-5. t}+0.0024 e^{5. t} \\
e^{-5. t} \left(0.012 e^{10. t}-0.012\right) \\
\end{bmatrix}
$$
\newpage
Plotting the response we get
\begin{figure}[!htb]
\centering
\includegraphics[scale=.75]{{images/p1_zero_input_response}.png}
\end{figure}
\newline
We can see that the system is unstable in all states.
This is only to $.2$ seconds as going further out quickly becomes a runaway state going to large orders of magnitude after leaving the linearized region.
The angle of the pendulum falls rapidly away from the vertical equilibrium increasing it's angular velocity which pushes the cart in the positive $x$ direction gaining speed.
This system is clearly not stable on it's own.

\item Now, use the state feedback $u(t) = Fx(t)$ with $F = \begin{bmatrix} 139 & 24 & 30 & 31 \end{bmatrix}$ to stabilize the system.
Find the closed-loop pole locations of the closed-loop system to verify that all these poles are in the left half of the complex plane.
\newline
\newline
Our system has the form $\dot{x} = Ax +Bu$ substituting in our feedback control we get our system as
$$\dot{x} = Ax + BFx = (A+BF)x$$
To check the stability we need to look at the eigenvalues of the new matrix $A+BF$.
\begin{align*}
A + BF &=
\left(
\begin{bmatrix}
0 & 1 & 0 & 0 \\
25 & 0 & 0 & 0 \\
0 & 0 & 0 & 1 \\
-0.4 & 0 & 0 & 0 \\
\end{bmatrix}
+
\begin{bmatrix}
0 \\
-1.4 \\
0 \\
0.6 \\
\end{bmatrix}
\begin{bmatrix} 139 & 24 & 30 & 31 \end{bmatrix}
\right)
x(t) \\
A + BF &=
\left(
\begin{bmatrix}
0 & 1 & 0 & 0 \\
25 & 0 & 0 & 0 \\
0 & 0 & 0 & 1 \\
-0.4 & 0 & 0 & 0 \\
\end{bmatrix}
+
\begin{bmatrix}
0 & 0 & 0 & 0 \\
-194.6 & -33.6 & -42 & -43.4 \\
0 & 0 & 0 & 0 \\
83.4 & 14.4 & 18 & 18.6 \\
\end{bmatrix}
\right)
x(t) \\
A + BF &=
\begin{bmatrix}
0 & 1. & 0 & 0 \\
-169.6 & -33.6 & -42 & -43.4 \\
0 & 0 & 0 & 1 \\
83 & 14.4 & 18 & 18.6 \\
\end{bmatrix}
x(t)
\end{align*}
Finding the eigenvalues of the matrix
$$
\begin{vmatrix}
\lambda & -1. & 0 & 0 \\
169.6 & \lambda+33.6 & 42 & 43.4 \\
0 & 0 & \lambda & -1 \\
-83 & -14.4 & -18 & \lambda -18.6 \\
\end{vmatrix}
= x^4 + 15 x^3 + x^2 + 447.64 x + 433.2 = 0
$$
Using Mathematica to solve the quartic polynomial we get the eigenvalues as
$$
\lambda_1 = -5.582 - 8.572i \quad \lambda_2 = -5.582 + 8.572i
\quad
\lambda_3 = -1.918 - 0.679i \quad \lambda_4 = -1.918 + 0.679i
$$
Which does have all eigenvalues in the left half plane, so the controlled system is stable.

\item Plot the displacement of the state variables $x_1(t)$ and $x_3(t)$ on the first graph, the velocity state variables $x_2(t)$ and $x_4(t)$ on the second graph, and the control input $u(t)$ on the third graph for the closed-loop system.
\newline
\newline
Using mathematica to plot $x(t) = e^{At}x_0$ and $u(t) = Fe^{At}x_0$ we get
\begin{figure}[!htb]
\centering
\includegraphics[scale=.5]{{images/p1_controlled_response}.png}
\end{figure}

\item Comment on the physical meaning of the three time response graphs in (d).
\newline
\newline
Assuming that positive displacement is to the right, positive rotation is clockwise, and positive input is force applied to the right.
We can see that the cart needs a large push to the left to stop the pendulum from falling over.
There is a point where the state variables are almost at the origin just from this initial push to the left, but now the system has a lot of energy in it $x_2,x_4$.
So the system needs to dissipate the energy by overshooting the equilibrium point for $x_1,x_3$ with the pendulum in a position where it is leaning in the direction the cart is moving.
During this the cart begins to slow down and move back towards the origin.
The overshoot we see in the state variables is required to dissipate the energy gained when getting the pendulum in a more stable position.
Once the energies have been dissipated the cart has passed the equilibrium so it moves very slowly back so that there is no more energy buildup.

\end{enumerate}

\newpage
\section*{Problem 2}
Consider the following block diagram
\begin{figure}[!htb]
\centering
\tikzset{
    block/.style = {draw, fill=white, rectangle, minimum height=1em, minimum width=1em},
    tmp/.style  = {coordinate, node distance=1cm},
    sum/.style= {draw, fill=white, circle, node distance=1cm},
    input/.style = {coordinate, node distance=1cm},
    output/.style= {coordinate, node distance=2cm},
    pinstyle/.style = {pin edge={to-,thin,black}},
}
\begin{tikzpicture}[auto, node distance=2cm,>=latex']
\node [input, name=input](input){};
\node [block, right of=input](transferFunc){$G(s)$};
\node [output, right of=transferFunc](output) {};

\draw [->] (input) -- node[above]{$U(s)$} node[below]{$u(t)$} (transferFunc) {};
\draw [->] (transferFunc) -- node[above]{$Y(s)$} node[below]{$y(t)$} (output) {};
\end{tikzpicture}
\end{figure}
where $G(s)$ is given as
$$ G(s) = \frac{s-2}{s(s^2+1)} $$
\begin{enumerate}[(a)]
\item Explain why the system is not BIBO stable based on the \textbf{definition} of BIBO stability, and find two bounded inputs that would cause the output to be unbounded.
Plot the output due to these two inputs.
\newline
\newline
A system is BIBO stable if $\int_{-\infty}^{\infty} \abs{h(t)}dt < B$ for some constant $B$ where $h(t)$ is the impulse response.
So if we compute $h(t)$ and show it's not absolutely integrable the system is not BIBO stable.
Taking the inverse Laplace transform of $G(s)$ gives us the impulse response.
\begin{align*}
G(s) &= \frac{s-2}{s(s^2+1)} \\
&= \frac{1}{s^2+1} - \frac{2}{s(s^2+1)} \\
\mathcal{L}^{-1}\{G(s)\} &= 2 \cos{t} + \sin{t} -2
\end{align*}
The $\cos$ and $\sin$ terms will be bounded due to their cyclic nature.
We can see that this is integral is not bounded due to the constant term that will go to infinity.
Therefore the system is not BIBO stable.
\newline
A step response has a Laplace transform of $\frac{1}{s}$
\begin{align*}
x_{\text{step}}(t) &= \mathcal{L}^{-1}\{G(s) \frac{1}{s}\} \\
 &= \mathcal{L}^{-1}\{ \frac{s-2}{s^2(s^2+1)} \} \\
 &= \mathcal{L}^{-1}\{ \frac{1}{s^2+1} - \frac{2}{s^2(s^2+1)} \} \\
 &= 2 \sin{t} - \cos{t} -2t + 1 \\
\end{align*}
Which is clearly unbounded as $ \lim_{x \to \infty}x(t) = -\infty $ due to the $-2t$ term.
\newline
A sinusoidal response has a Laplace transform of $\frac{1}{s^2+1}$
\begin{align*}
x_{\text{step}}(t) &= \mathcal{L}^{-1}\{G(s) \frac{1}{s^2+1}\} \\
 &= \mathcal{L}^{-1}\{ \frac{s-2}{s(s^2+1)^2} \} \\
 &= \mathcal{L}^{-1}\{ \frac{1}{(s^2+1)^2} - \frac{2}{s(s^2+1)^2} \} \\
 &= \sin{(t)}(\frac{1}{2} + t)-\cos{(t)}(\frac{1}{2}t -2) - 2 \\
 &= \frac{1}{2}(\sin{t} + 4\cos{t} -4) + \frac{1}{2}t(-\cos{t} +2\sin{t})
\end{align*}
Which is clearly unbounded as $ \lim_{x \to \infty}x(t) $ oscillates growing larger and larger due to the $\frac{1}{2}t(-\cos{t} +2\sin{t})$ term.
\newpage
\begin{figure}[!htb]
\centering
\includegraphics[scale=.5]{{images/p2_2_unbouded_responses}.png}
\caption*{2 Responses that are unbounded from bounded inputs}
\end{figure}

\item Find a state-space representation in the controller canonical form for the system $G(s)$.
\begin{align*}
\dot{x}(t) &= Ax(t) + Bu(t) \\
y(t) &= Cx(t) + Du(t)
\end{align*}
\newline
Following the formula in the notes for the controller canonical form we get the state space representation as
\begin{align*}
\dot{x}(t)
&=
\begin{bmatrix}
0 &  1 & 0 \\
0 &  0 & 1 \\
0 & -1 & 0 \\
\end{bmatrix}
x(t)
+
\begin{bmatrix}
0 \\
0 \\
1 \\
\end{bmatrix}
u(t)
\\
y(t)
&=
\begin{bmatrix}
-2 \\
 1 \\
 0 \\
\end{bmatrix}
x(t)
+
\begin{bmatrix}
0
\end{bmatrix}
u(t)
\end{align*}
With nothing interesting happening in the $C$ matrix as $b_0 = 0$


\item Explain why the system is not internally stable based on the \textbf{definition} of internal stability.
Plot the state response, $x(t)$, of the system with initial state $x(0) = \begin{bmatrix}2 & 0 & -2 \end{bmatrix}^T$ and zero input.
\newline
\newline
By the definition of internal stability we mush show that $\dot{x}(t) = Ax(t)$ \textit{always} tends towards zero as $t \to \infty$.
So to show that it's not internally stable we just need to show that for some, or all, inputs we don't tend towards zero.
As in problem 1 the solution to this equation is just $x(t) = e^{A t}$.
The eigenvalues for this system are $\pm i, 0$, since non are repeated we can diagonalize the system.
Using $e^{A t} = Pe^{Dt}P^{-1}$ we get
$$
P =
\begin{bmatrix}
-1 & -1 & 1 \\
-i & i & 0 \\
1 & 1 & 0 \\
\end{bmatrix}
\quad
D =
\begin{bmatrix}
i & 0 & 0 \\
0 & i & 0 \\
0 & 0 & 0 \\
\end{bmatrix}
\implies
e^{A t} =
\begin{bmatrix}
1 & \sin{t} & 1-\cos{t} \\
0 & \cos{t} & \sin{t} \\
0 & -\sin{t} & \cos{t} \\
\end{bmatrix}
$$
Multiplying this by any matrix we get that all of our internal states are some combination of sinusoids.
Due to this the internal states will never stop oscillating.
Therefore the system is internally unstable.
$$
\begin{bmatrix}
1 & \sin{t} & 1-\cos{t} \\
0 & \cos{t} & \sin{t} \\
0 & -\sin{t} & \cos{t} \\
\end{bmatrix}
\begin{bmatrix}
2 \\
0 \\
-2 \\
\end{bmatrix}
=
\begin{bmatrix}
2\cos{t} \\
-2\sin{t} \\
-2\cos{t} \\
\end{bmatrix}
$$
\begin{figure}[!htb]
\centering
\includegraphics[scale=.4]{{images/p2_zero_impulse_response}.png}
\end{figure}

\end{enumerate}

\newpage
\section*{Problem 3}
Consider the following feedback control system.
\begin{figure}[!htb]
\centering
\tikzset{
    block/.style = {draw, fill=white, rectangle, minimum height=1em, minimum width=1em},
    tmp/.style  = {coordinate, node distance=1cm},
    sum/.style= {draw, fill=white, circle, node distance=1cm},
    input/.style = {coordinate, node distance=1cm},
    output/.style= {coordinate, node distance=2cm},
    pinstyle/.style = {pin edge={to-,thin,black}},
}
\begin{tikzpicture}[auto, node distance=2cm,>=latex']
\node [input, name=input](input){};
\node [sum, right of=input, label=south east:{$-$}](inputSum){};
\node [block, right of=inputSum](controller){$K(s)$};
\node [block, right of=controller](transferFunc){$G(s)$};
\node [tmp, right of=transferFunc](tmpOutput) {};
\node [output, right of=tmpOutput](output) {};
\node [tmp, below of=transferFunc](belowTF) {};

\draw [->] (input) -- node[above]{$R(s)$} node[below]{$r(t)$} (inputSum) {};
\draw [->] (inputSum) --  (controller) {};
\draw [->] (controller) -- (transferFunc) {};
\draw [-] (transferFunc) -- (tmpOutput) {};
\draw [->] (tmpOutput) -- node[above]{$Y(s)$} node[below]{$y(t)$} (output) {};
\draw [-] (tmpOutput) |- (belowTF) {};
\draw [->] (belowTF) -| (inputSum) {};
\end{tikzpicture}
\end{figure}
Given
$$G(s) = \frac{s-1}{s^2-2s}$$

\begin{enumerate}[(a)]
\item Find a controller $K(s)$ with the least order, which is either strictly proper or proper, so that the closed-loop system is BIBO stable.
Hint: You can start from a zero-order controller $K_0(s) = b_0$. If it does not work, try a $1^{\text{st}}$-order controller $ K_1(s) = \frac{b_1 s + b_0}{s+a_0}$ or a $2^{\text{nd}}$-order controller $K_2(s) = \frac{b_2 s^2 + b_1 s + b_0}{s^2 + a_1 s + a_0}$.
\newline
\newline
Assuming that we have a perfect model we can perform pole cancellation exactly.
We need to remove both poles since they are $0$ and $2$ which are not strictly negative.
\newline
Using the second order controller since we need to cancel 2 poles
$$ K(s) = \frac{s^2-2s}{(s+1)(s+3)} = \frac{s^2-2s}{s^2+4s+3} $$

Computing this here cause we will use it later
$$ K(s) G(s) = \frac{(s-1)}{(s+1)(s+3)} $$

\item With the controller $K(s)$ you have just designed, determine the closed-loop transfer function $\frac{Y(s)}{R(s)}$.
\newline
\newline
Let $Z(s) = R(s) - Y(s)$. Then we have $Z(s) K(s) G(s) = Y(s)$.
Substituting we get
\begin{align*}
Y(s) &= (R(s) - Y(s))K(s) G(s) \\
Y(s) &= R(s)K(s) G(s) - Y(s)K(s) G(s) \\
Y(s) + Y(s)K(s) G(s)  &= R(s)K(s) G(s) \\
Y(s)(1 + K(s) G(s))  &= R(s)K(s) G(s) \\
\frac{Y(s)}{R(s)}  &= \frac{K(s) G(s)}{1 + K(s) G(s)} \\
\end{align*}
Plugging in our values we get
$$ \frac{Y(s)}{R(s)} = \frac{\frac{(s-1)}{(s+1)(s+3)}}{1 + \frac{(s-1)}{(s+1)(s+3)}} $$
Using Mathematica to simplify we find
$$ \frac{Y(s)}{R(s)} = \frac{s-1}{s^2+5s + 2} $$
Which has poles
$$ \frac{1}{2}(-5 \pm \sqrt{17}) $$

\item Assume zero initial conditions and the reference input $r(t)$ is a unit step function.
Plot the output response $y(t)$ of the feedback control system you have just designed.
\newline
\newline
A unit step is $\frac{1}{s}$ in the Laplace domain.
$$ Y_{\text{step}}(s) = \frac{s-1}{s^2+5s + 2} \frac{1}{s} $$
$$ Y_{\text{step}}(s) = \frac{s-1}{s(s^2+5s + 2)} $$
$$ Y_{\text{step}}(s) = \frac{s-1}{s(s^2+5s + 2)} $$
$$ Y_{\text{step}}(s) = \frac{1}{s^2 + 5s + 2} - \frac{1}{s(s^2+5s+2)} $$
$$ y_{\text{step}}(t) = \frac{1}{68} e^{-\frac{1}{2} \left(5+\sqrt{17}\right) t} \left(\left(17+9 \sqrt{17}\right) e^{\sqrt{17} t}-34e^{\frac{1}{2} \left(5+\sqrt{17}\right) t}+17-9 \sqrt{17}\right) $$

\begin{figure}[!htb]
\centering
\includegraphics[scale=.5]{{images/p3_step_response}.png}
\end{figure}

\end{enumerate}

\newpage
\section*{Problem 4}
Consider the system described by the following state equation,
$$ \dot{x}(t) =
\begin{bmatrix}
-21 & -22 & -20 \\
26 & 27 & 23 \\
-9 & -9 & -7 \\
\end{bmatrix}
x(t) +
\begin{bmatrix}
-3 \\
5 \\
-2 \\
\end{bmatrix}
u(t)
$$

\begin{enumerate}[(a)]
\item Use PBH Test to check if the system is controllable.
\newline
\newline
A system is controllable if the $rank\left(\begin{bmatrix} sI - A & b \end{bmatrix}\right) = n \quad \forall s \in \mathcal{C}$
$$
\begin{bmatrix}
s + 21 & 22 & 20 & -3 \\
-26 & s - 27 & -23 & 5 \\
9 & 9 & s + 7  & -2 \\
\end{bmatrix}
$$
If the $rank\left(A\right) = n$ then the only times when the first entry $sI -A$ will not have $rank = n$ will be when $s$ is equal to an eigenvalue.
So if we check that the determinant of $A$ is nonzero we know $A$ is of full rank.
Then we need to find the eigenvalues and test the matrix above letting $s = \lambda_i \quad i \in \{1, ... n\}$
$$
\begin{vmatrix}
\lambda + 21 & 22 & 20 \\
-26 & \lambda - 27 & -23 \\
9 & 9 & \lambda + 7 \\
\end{vmatrix}
= -\lambda^3 - \lambda^2 + 10\lambda -8
$$
Since this isn't always zero we the $A$ matrix is full rank and we can proceed with finding the eigenvalues.
$$ \lambda_1 = -4 \quad \lambda_2 = 2 \quad \lambda_3 = 1 $$
Plugging in each eigenvalue we can check if the matrix is still rank 3
$$
\begin{bmatrix}
-4 + 21 & 22 & 20 & -3 \\
-26 & -4 - 27 & -23 & 5 \\
9 & 9 & -4 + 7  & -2 \\
\end{bmatrix}
=
\begin{bmatrix}
17 & 22 & 20 & -3 \\
-26 & -23 & -23 & 5 \\
9 & 9 & 3  & -2 \\
\end{bmatrix}
$$
Performing row reduction in Mathematica we get
$$
\begin{bmatrix}
1 & 0 & -\frac{38}{15} & -\frac{17}{45} \\
0 & 1 &  \frac{43}{15} &  \frac{7}{45} \\
0 & 0 & 0 & 0 \\
\end{bmatrix}
$$
Which shows that the matrix has rank 2.
So by the PBH this system is uncontrollable.
\newline
\newline
By the same process we can check the other eigenvalues
$$ rank(s=2) = 3 \quad rank(s=1) = 3 $$

\item Characterize the controllable subspace.
\newline
\newline
The pbh test told us that the eigenvalue $\lambda = -4$ is uncontrollable which tells us that in the direction of that eigenvector we have no control.
Since the other eigenvalues had full rank we are controllable in their directions.
If we are non-controllable in 1 dimension while in $\mathcal{R}^3$ then the controllable subspace is in the plane which contains the eigenvectors corresponding to the eigenvalues $\lambda = 1$ and $\lambda = 2$.

\item If possible, design a state feedback controller so that the closed-loop system is internally stable.
If not, explain the reason.
\newline
\newline

\end{enumerate}

\end{document}
