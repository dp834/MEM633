\documentclass{article}

\usepackage[margin=.75in]{geometry}
\usepackage{amsmath}
\usepackage{amssymb}
\usepackage[shortlabels]{enumitem}
\usepackage{pgfplots}
\usepackage{float}
\usepackage{verbatim}
\usepackage{caption}
\usepackage{subcaption}

\author{Anthony Siddique, Damien Prieur, Jachin Philip, Obinna Ekeh}
\title{Homework 1\\ MEM 633 \\ Group 1}
\date{}

\begin{document}

\maketitle

\section*{Problem 1}
Given: A system is governed by
\begin{align*}
\ddot{r} &= r\dot{\theta}^2 - \frac{k}{r^2} + u_1 \\
\ddot{\theta} &= \frac{-2\dot{\theta}\dot{r}}{r}+\frac{1}{r}u_2
\end{align*}

If $u_1 = u_2 = 0$, the equations admit the solution
\begin{align*}
r &= \sigma \quad (\sigma \text{is a constant})\\
\theta &= \omega t \quad (\omega \text{is a constant})\\
\end{align*}
where $\sigma^3\omega^2 = k$.
Define
$x_1 = r - \sigma$,
$x_2 = \dot{r}$,
$x_3 = \sigma(\theta-\omega t)$,
$x_4 = \sigma(\dot{\theta} - \omega)$,
and write a set of linearized differential equations which describe the resulting motion for small deviations from a given circular orbit.
\newline
\newline
$$
\begin{bmatrix}
x_1\\
x_2\\
x_3\\
x_4\\
\end{bmatrix}
=
\begin{bmatrix}
r - \sigma\\
\dot{r}\\
\sigma(\theta-\omega t)\\
\sigma(\dot{\theta} - \omega)\\
\end{bmatrix}
$$
$$
\begin{bmatrix}
\dot{x}_1\\
\dot{x}_2\\
\dot{x}_3\\
\dot{x}_4\\
\end{bmatrix}
=
\begin{bmatrix}
\dot{r}\\
\ddot{r}\\
\sigma(\dot{\theta}-\omega)\\
\sigma\ddot{\theta}\\
\end{bmatrix}
=
\begin{bmatrix}
x_2\\
r\dot{\theta}^2 - \frac{\sigma^3\omega^2}{r^2} \\
x_4\\
\sigma\frac{-2\dot{\theta}\dot{r}}{r}\\
\end{bmatrix}
$$
Since we want these equations in terms of $x_1,x_2,x_3,x_4$ we can solve for $r,\dot{r},\theta,\dot{\theta}$ and substitute.

$$
\begin{bmatrix}
\dot{x}_1\\
\dot{x}_2\\
\dot{x}_3\\
\dot{x}_4\\
\end{bmatrix}
=
\begin{bmatrix}
x_2 \\
(x_1 + \sigma)(\frac{x_4}{\sigma} + \omega)^2 - \frac{\sigma^3\omega^2}{(x_1 + \sigma)^2} \\
x_4 \\
\sigma \frac{-2(\frac{x_4}{\sigma} + \omega)x_2}{x_1+\sigma} \\
\end{bmatrix}
$$
Linearizing the system we get
$$
\begin{bmatrix}
\dot{x}_1\\
\dot{x}_2\\
\dot{x}_3\\
\dot{x}_4\\
\end{bmatrix}
=
\begin{bmatrix}
0 & 1 & 0 & 0 \\
\frac{(2\omega^2\sigma^3)}{(\sigma + x_1)^3} + (\omega + \frac{x_4}{\sigma})^2 & 0 & 0 & \frac{2(\sigma + x_1)(\omega\sigma + x_4)}{\sigma^2} \\
0 & 0 & 0 & 1 \\
\frac{2 x_2 (\omega\sigma + x_4)}{(\sigma + x_1)^2} & \frac{-2(\omega\sigma + x_4)}{\sigma + x_1} & 0 & \frac{-2 x_2}{\sigma + x_1} \\
\end{bmatrix}
$$
Evaluating at our equilibrium point we get
\begin{align*}
x_1 = 0 \\
x_2 = 0 \\
x_3 = 0 \\
x_4 = 0 \\
\end{align*}
And plugging these into the linearized model we find
$$
\begin{bmatrix}
\dot{x}_1\\
\dot{x}_2\\
\dot{x}_3\\
\dot{x}_4\\
\end{bmatrix}
=
\begin{bmatrix}
0 & 1 & 0 & 0 \\
3 \omega^2 & 0 & 0 & 2\omega \\
0 & 0 & 0 & 1 \\
0 & -2\omega & 0 & 0 \\
\end{bmatrix}
$$
which describes the motion of the system for small deviations from a given circular orbit defined by the solution $r = \sigma$ and $\theta = \omega t$.

\section*{Problem 2}
Given: A system is governed by
\begin{align*}
\dot{v}_1+5v_1 &= 2u \\
\dot{v}_2+v_2 -2v_1 &= 0 \\
y &= 50 v_2
\end{align*}
where $v_1(t)$ and $v_2(t)$ are state variables and $y(t)$ and $u(t)$ are the output and the input of the system respectively.
\begin{enumerate}[(a)]
\item Compute the state transition matrix, the impulse response function, and the step response function of the system.
Plot the impulse and the step responses based on the obtained functions. Is the system stable? Why?
\newline
\newline
We can begin by writing the system in the form $\dot{x} = Ax +Bu$ and $y = Cx + Du$.
Doing this we get
$$
\begin{bmatrix}
\dot{v}_1 \\
\dot{v}_2 \\
\end{bmatrix}
=
\begin{bmatrix}
-5 & 0 \\
2 & -1 \\
\end{bmatrix}
\begin{bmatrix}
v_1 \\
v_2 \\
\end{bmatrix}
+
\begin{bmatrix}
2 \\
0 \\
\end{bmatrix}
\begin{bmatrix}
u
\end{bmatrix}
$$
$$
y =
\begin{bmatrix}
0 & 50 \\
\end{bmatrix}
\begin{bmatrix}
v_1 \\
v_2 \\
\end{bmatrix}
$$
To find the state transition matrix we compute $e^{A (t-t_0)}$
$$
\Phi(t,t_0) =
\begin{bmatrix}
-e^{-2 t + t_0} (e^{t_0}-2 e^t) & e^{-t + t_0} (e^t-e^{t_0}) \\
 2 e^{-t + t_0} (e^{t_0}-e^t) & e^{-2t + t_0} (2 e^{t_0}-e^t) \\
\end{bmatrix}
$$
The solution to $x(t)$ is
$$ x(t) = e^{A t} x(t_0) + \int_0^t \Phi(t,\tau)B(\tau)u(\tau)d\tau $$
For the impulse response we have $u(t) = \delta(t)$ and using the property of the delta function
$$ \int_{-\infty}^\infty f(t)\delta(t-T) dt = f(T) $$
Then the impulse response of our system is
$$ x(t) = e^{A t}x(t_0)  + \Phi(t,0)B $$
$$ x(t) = e^{A t}x(t_0)  + e^{A t}B $$
$$
\begin{bmatrix}
v_1(t) \\
v_2(t) \\
\end{bmatrix}
=
\begin{bmatrix}
e^{-5 t} (x_{10}+2) \\
\frac{1}{2} e^{-5 t} (e^{4 t} (x_{10}+2 x_{20}+2)-x_{10}-2) \\
\end{bmatrix}
$$
Where $x(0) = \begin{bmatrix} v_{10} \\ v_{20} \end{bmatrix}$
Using $(0,0)$ as the initial conditions we get
$$
\begin{bmatrix}
v_1(t) \\
v_2(t) \\
\end{bmatrix}
=
\begin{bmatrix}
2 e^{-5 t} \\
e^{-5 t} (e^{4 t}-1) \\
\end{bmatrix}
$$
Now just for the output of the system we get
$$ y(t) = 50e^{-5 t} (e^{4 t}-1) $$
\begin{figure}[h!]
\centering
\includegraphics[scale=.75]{{images/p2_impulse}.png}
\end{figure}


Now to find the step response we solve
$$ x(t) = e^{A t} x(t_0) + \int_0^t \Phi(t,\tau)B(\tau)u(\tau)d\tau $$
Where $u(t) = 1$
$$ x(t) = e^{A t} x(t_0) + \int_0^t \Phi(t,\tau)B d\tau $$
Solving this we get
$$
\begin{bmatrix}
v_1(t) \\
v_2(t) \\
\end{bmatrix}
=
\begin{bmatrix}
\frac{1}{5} (e^{-5 t} (5 x_{10}-2)+2) \\
\frac{1}{10} e^{-5 t} (5 e^{4 t} (x_{10}+2 x_{20}-2)+8 e^{5 t}-5 x_{10}+2) \\
\end{bmatrix}
$$
Again, using $(0,0)$ as the initial conditions we get
$$
\begin{bmatrix}
v_1(t) \\
v_2(t) \\
\end{bmatrix}
=
\begin{bmatrix}
\frac{1}{5} (2-2 e^{-5 t}) \\
\frac{1}{5} (e^{-5 t}-5 e^{-t}+4) \\
\end{bmatrix}
$$
Again just for the output of the system we get
$$ y(t) = 10 (e^{-5 t}-5 e^{-t}+4) $$
\begin{figure}[h!]
\centering
\includegraphics[scale=.75]{{images/p2_step}.png}
\end{figure}

For stability we can look at the eigenvalues of $A$
$$det(A-\lambda I) = 0$$
$$
\begin{vmatrix}
-5-\lambda & 0 \\
2 & -1 - \lambda
\end{vmatrix}
=(\lambda+5)(\lambda+1) \implies \lambda_1 = -5 \qquad \lambda_2 = -1
$$
Since both of our eigenvalues are in the left half plane the system is stable.



\item Determine the transfer function of the system.
Draw the Bode plots of the frequency response function of the system, and explain the physical meaning of the plots.
\newline
\newline
The transfer function can be obtained by computing
$$ H(s) = \frac{Y(s)}{U(s)} = C(sI-A)^{-1}B + D $$
Starting with $(sI-A)^{-1}$
$$ (sI-A)^{-1} =
\begin{bmatrix}
s+5 & 0 \\
-2 & s+1\\
\end{bmatrix}
^{-1}
=
\frac{1}{(s+5)(s+1)}
\begin{bmatrix}
s+1 & 0 \\
2  & s+5 \\
\end{bmatrix}
=
\begin{bmatrix}
\frac{1}{s+5} & 0 \\
\frac{2}{(s+5)(s+1)} & \frac{1}{s+1} \\
\end{bmatrix}
$$
Computing $H(s)$ we get
$$ H(s) = \frac{200}{s^2 + 6s +5} $$
To create the bode plot we substitute $ s = \omega j$ and plot $|H(s)|$
$$ |H(\omega j)| = |\frac{200}{-\omega^2 + 5 + 6\omega j }| $$
$$ |H(\omega j)| = \frac{200}{\sqrt{(\omega ^2+1) (\omega ^2+25)}} $$
Graphing and computing the phase and gain margins we find
\newline
\begin{figure}[H]
\centering
\includegraphics[scale=.75]{{images/p2_bode_plot}.png}
\end{figure}
We have a gain margin of $\sqrt{4 \sqrt{2509}-13} \approx 13.69$ and a phase margin of $\tan ^{-1}\left(\frac{6 \sqrt{4 \sqrt{2509}-13}}{4 \sqrt{2509}-18}\right) \approx .42$
Which means our system will remain stable if we increase our gain up to $13.7$, and our phase can be up to $.42$ radians and still remain stable.
This doesn't guarantee they can both change that amount and still be stable.
\end{enumerate}


\end{document}
