\documentclass{article}

\usepackage[margin=.75in]{geometry}
\usepackage{amsmath}
\usepackage{amssymb}
\usepackage[shortlabels]{enumitem}
\usepackage{pgfplots}
\usepackage{float}
\usepackage{verbatim}
\usepackage{caption}
\usepackage{subcaption}

\usepackage{mathtools}
\DeclarePairedDelimiter\norm{\lVert}{\rVert}%

\author{Anthony Siddique, Damien Prieur, Jachin Philip, Obinna Ekeh}
\title{Homework 2\\ MEM 633 \\ Group 1}
\date{}

\begin{document}

\maketitle

\section*{Problem 1}
$$ A = T \Lambda T^{-1} \qquad \Lambda = diag(\lambda_1,\lambda_2, ..., \lambda_n) $$
Where $\lambda_i$ are distinct.
Show that
\begin{enumerate}[(a)]
\item $e^{At} = Te^{\Lambda t}T^{-1}$
\newline
It is given that
$$ e^x = \sum_{i=0}^\infty \frac{x^i}{i!} $$
Plugging in $At$ we have
$$ e^{At} = \sum_{i=0}^\infty \frac{(At)^i}{i!} $$
Since $t$ is a scalar and $A$ is not it can be helpful to group like terms together
$$ e^{At} = \sum_{i=0}^\infty A^i\frac{t^i}{i!} $$
Now we plug in our definition of $A = T \Lambda T^{-1} $
$$ e^{At} = \sum_{i=0}^\infty (T \Lambda T^{-1})^i\frac{t^i}{i!} $$
TODO: Show that the matrix power reduces to $(T \Lambda^i T^{-1}$ and factor the $T$'s
The powers of $A$ can be expanded as such
$$ A^n = T\Lambda T^{-1} T\Lambda T^{-1}...T\Lambda T^{-1} = T\Lambda^n T^{-1} $$
All of the $T$'s and $T^{-1}$ are paired off expect the first $T$ and the last $T^{-1}$.
Plugging this into our formula we get
$$ e^{At} = \sum_{i=0}^\infty T \Lambda^i T^{-1}\frac{t^i}{i!} $$
Each term of the sum as a $T$ on the left and a $T^{-1}$ on the right so we can factor them out.
$$ e^{At} = T(\sum_{i=0}^\infty \Lambda^i \frac{t^i}{i!})T^{-1} $$
By definition the sum $\sum_{i=0}^\infty \Lambda^i \frac{t^i}{i!}$ is equal to the matrix exponential $e^{\Lambda t}$.
Substituting that in we get
$$ e^{At} = Te^{\Lambda t}T^{-1} $$



\item $e^{\Lambda t} = diag (e^{\lambda_1 t}, e^{\lambda_2 t}, e^{\lambda_n t})$
\newline
A the multiplication of two diagonal matricies  $diag(a_1, a_2, ..., a_n) diag(b_1, b_2, ..., b_n)$ is another diagonal matrix with each entry multiplied by the corresponding entry $ diag(a_1b_1 a_2b_2, ..., a_nb_n)$
So we have
$$ e^{\Lambda t} = \sum_{i=0}^\infty \frac{(\Lambda t) ^i}{i!} $$
Where each matrix will be a diagonal matrix of with $i$th powers of the eigenvalues.
Looking at the sum we can look at each diagonal entry as such.
$$ e^{\Lambda t}_k = \sum_{i=0}^\infty \frac{(\lambda_k t) ^i}{i!} $$
Where $e^{\Lambda t}_k$ represents the value of the $k$th eigenvalue or diagonal entry.
We can see that this is by defenintion equal to
$$ e^{\lambda_k t} = \sum_{i=0}^\infty \frac{(\lambda_k t) ^i}{i!} $$
So the $k$th entry on the diagonal of $e^{\Lambda t}$ will be $e^{\lambda t}$
Putting it all together we get
$$ e^{\Lambda t} = diag(e^{\lambda_1 t}, e^{\lambda_2 t}, ... , e^{\lambda_n t})$$

\end{enumerate}

\section*{Problem 2}
Consider the time-invariant system $\dot{x}(t) = Ax(t)$ where the $n$x$n$ matrix $A$ has distinct eigenvalues $\lambda_1, \lambda_2, ..., \lambda_n$.
The corresponding eigenvectors are $e_1, e_2, ...,e_n$.
Let $T = [e_1,e_2,...,e_n]$ and $v_1', v_2',...,v_n'$ be the row vectors of $T^{-1}$.
Show that the solution of $\dot{x}(t)=Ax(t)$ can be written as
$$ x(t) = \sum_{i=1}^n v_i' x(0) e^{\lambda_i t} e_i $$
\newline
\newline

\section*{Problem 3}
The $A$ matrix in Problem 2 is given as
$$ A =
\begin{bmatrix}
0 & 1 \\
6 & -5 \\
\end{bmatrix}
$$
Write down the solution of $\dot{x}(t) = Ax(t)$ by using the result of Problem 2.
You will see the system is unstable.
However, $x(t)$ will be bounded if the initial state vector is in the stable subspace.
Describe the stable subspace of the system.
\newline


\section*{Problem 4}
Find the realizations in controller and observability forms of the transfer function
$$ H(s) = \frac{2s^3+13s^2+31s+32}{s^3+6s^2+11s+6} $$
Give both block diagrams and state-space equations.


\end{document}
