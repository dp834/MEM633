\documentclass{article}

\usepackage[margin=.75in]{geometry}
\usepackage{amsmath}
\usepackage{amssymb}
\usepackage[shortlabels]{enumitem}
\usepackage{float}
\usepackage{verbatim}
\usepackage{caption}
\usepackage{subcaption}
\usepackage{tikz}
\usetikzlibrary{shapes,arrows,positioning,calc}

\usepackage{mathtools}
\DeclarePairedDelimiter\norm{\lVert}{\rVert}%

\author{Anthony Siddique, Damien Prieur, Jachin Philip, Obinna Ekeh}
\title{Homework 3\\ MEM 633 \\ Group 1}
\date{}

\begin{document}

\maketitle

\section*{Problem 1}
$$ A = T \Lambda T^{-1} \qquad \Lambda = diag(\lambda_1,\lambda_2, ..., \lambda_n) $$
Where $\lambda_i$ are distinct.
Show that
\begin{enumerate}[(a)]
\item $e^{At} = Te^{\Lambda t}T^{-1}$
\newline
It is given that
$$ e^x = \sum_{i=0}^\infty \frac{x^i}{i!} $$
Plugging in $At$ we have
$$ e^{At} = \sum_{i=0}^\infty \frac{(At)^i}{i!} $$
Since $t$ is a scalar and $A$ is not it can be helpful to group like terms together
$$ e^{At} = \sum_{i=0}^\infty A^i\frac{t^i}{i!} $$
Now we plug in our definition of $A = T \Lambda T^{-1} $
$$ e^{At} = \sum_{i=0}^\infty (T \Lambda T^{-1})^i\frac{t^i}{i!} $$
The powers of $A$ can be expanded as such
$$ A^n = T\Lambda T^{-1} T\Lambda T^{-1}...T\Lambda T^{-1} = T\Lambda^n T^{-1} $$
All of the $T$'s and $T^{-1}$ are paired off expect the first $T$ and the last $T^{-1}$.
Plugging this into our formula we get
$$ e^{At} = \sum_{i=0}^\infty T \Lambda^i T^{-1}\frac{t^i}{i!} $$
Each term of the sum as a $T$ on the left and a $T^{-1}$ on the right so we can factor them out.
$$ e^{At} = T(\sum_{i=0}^\infty \Lambda^i \frac{t^i}{i!})T^{-1} $$
By definition the sum $\sum_{i=0}^\infty \Lambda^i \frac{t^i}{i!}$ is equal to the matrix exponential $e^{\Lambda t}$.
Substituting that in we get
$$ e^{At} = Te^{\Lambda t}T^{-1} $$



\item $e^{\Lambda t} = diag (e^{\lambda_1 t}, e^{\lambda_2 t}, e^{\lambda_n t})$
\newline
The multiplication of two diagonal matrices  $diag(a_1, a_2, ..., a_n) diag(b_1, b_2, ..., b_n)$ is another diagonal matrix with each entry multiplied by the corresponding entry $ diag(a_1b_1 a_2b_2, ..., a_nb_n)$
So we have
$$ e^{\Lambda t} = \sum_{i=0}^\infty \frac{(\Lambda t) ^i}{i!} $$
Where each matrix will be a diagonal matrix of with $i$th powers of the eigenvalues.
Looking at the sum we can look at each diagonal entry as such.
$$ e^{\Lambda t}_k = \sum_{i=0}^\infty \frac{(\lambda_k t) ^i}{i!} $$
Where $e^{\Lambda t}_k$ represents the value of the $k$th eigenvalue or diagonal entry.
We can see that this is by definition equal to
$$ e^{\lambda_k t} = \sum_{i=0}^\infty \frac{(\lambda_k t) ^i}{i!} $$
So the $k$th entry on the diagonal of $e^{\Lambda t}$ will be $e^{\lambda t}$
Putting it all together we get
$$ e^{\Lambda t} = diag(e^{\lambda_1 t}, e^{\lambda_2 t}, ... , e^{\lambda_n t})$$

\end{enumerate}

\newpage
\section*{Problem 2}
Consider the time-invariant system $\dot{x}(t) = Ax(t)$ where the $n$x$n$ matrix $A$ has distinct eigenvalues $\lambda_1, \lambda_2, ..., \lambda_n$.
The corresponding eigenvectors are $e_1, e_2, ...,e_n$.
Let $T = [e_1,e_2,...,e_n]$ and $v_1', v_2',...,v_n'$ be the row vectors of $T^{-1}$.
Show that the solution of $\dot{x}(t)=Ax(t)$ can be written as
$$ x(t) = \sum_{i=1}^n v_i' x(0) e^{\lambda_i t} e_i $$
\newline
\newline
If we take the differential equation and take the Laplace transform $\mathbb{L}$ we get
$$
\begin{aligned}
\mathbb{L}(\dot{x}(t)) &= \mathbb{L}(Ax(t)) \\
sX(s) - x(0) &= AX(s) \\
(s-A)X(s) &= x(0) \\
X(s) &= (s-A)^{-1}x(0) \\
\end{aligned}
$$
Taking the inverse Laplace transform $\mathbb{L}^{-1}$ we can find the solution of $x(t)$
$$
\begin{aligned}
\mathbb{L}^{-1}(X(s)) &= \mathbb{L}^{-1}((s-A)^{-1}x(0)) \\
x(t) &= e^{At}x(0) \\
x(t) &= Te^{\Lambda t}T^{-1}x(0) \\
\end{aligned}
$$
Substituting $T$ and $T^{-1}$ we have
$$ x(t) =
\begin{bmatrix} e_1& e_2& ... & e_n\end{bmatrix}
diag(e^{\lambda_1 t},e^{\lambda_2 t}, ..., e^{\lambda_n t})
\begin{bmatrix} v_1' \\ v_2' \\ ... \\ v_n'\end{bmatrix}
x(0)
$$
Performing the multiplication on the left we get each $\lambda_i$ multiplied by it's corresponding $e_i$.
And performing the multiplication on the right we get $x(0)v_i'$
$$ x(t) =
\begin{bmatrix} e^{\lambda_1 t}e_1 & e^{\lambda_2 t}e_2 & ... & e^{\lambda_n t}e_n\end{bmatrix}
\begin{bmatrix} v_1'x(0) \\ v_2'x(0) \\ ... \\ v_n'x(0)\end{bmatrix}
$$
Writing this product as a general sum we get the form we are expecting
$$ x(t) = \sum_{i=1}^n v_i' x(0) e^{\lambda_i t} e_i $$

\newpage
\section*{Problem 3}
The $A$ matrix in Problem 2 is given as
$$ A =
\begin{bmatrix}
0 & 1 \\
6 & -5 \\
\end{bmatrix}
$$
Write down the solution of $\dot{x}(t) = Ax(t)$ by using the result of Problem 2.
You will see the system is unstable.
However, $x(t)$ will be bounded if the initial state vector is in the stable subspace.
Describe the stable subspace of the system.
\newline
We must first diagonalize the $A$ matrix.
Using Mathematica we find
$$
\begin{bmatrix}
0 & 1 \\
6 & -5 \\
\end{bmatrix}
=
\begin{bmatrix}
 -1 & 1 \\
  6 & 1 \\
\end{bmatrix}
\begin{bmatrix}
 -6 & 0 \\
  0 & 1 \\
\end{bmatrix}
\begin{bmatrix}
 -\frac{1}{7} & \frac{1}{7} \\
  \frac{6}{7} & \frac{1}{7} \\
\end{bmatrix}
$$
We showed in the last problem that the sum is equivalent to the matrix product
$ Te^{\Lambda t}T^{-1} x(0)$
With this we get
$$
x(t) =
\begin{bmatrix}
 -1 & 1 \\
  6 & 1 \\
\end{bmatrix}
\begin{bmatrix}
 e^{-6 t} & 0 \\
  0 & e^t \\
\end{bmatrix}
\begin{bmatrix}
 -\frac{1}{7} & \frac{1}{7} \\
  \frac{6}{7} & \frac{1}{7} \\
\end{bmatrix}
\begin{bmatrix}
x_1(0) \\
x_2(0) \\
\end{bmatrix}
=
\begin{bmatrix}
\left(\frac{6 e t}{7}+\frac{t}{7 e^6}\right) x_1(0)+\left(\frac{e t}{7}-\frac{t}{7 e^6}\right) x_2(0) \\
\left(\frac{6 e t}{7}-\frac{6 t}{7 e^6}\right) x_1(0)+\left(\frac{e t}{7}+\frac{6 t}{7 e^6}\right) x_2(0) \\
\end{bmatrix}
$$
The only way this remains bounded is if $x(t) = 0 \quad \forall t \in \mathbb{R}$.
So we want to find the nullspace of the matrix
$$
t
\begin{bmatrix}
\left(\frac{6 e }{7}+\frac{1}{7 e^6}\right) & \left(\frac{e}{7}-\frac{1}{7 e^6}\right) \\
\left(\frac{6 e }{7}-\frac{6}{7 e^6}\right) & \left(\frac{e}{7}+\frac{6}{7 e^6}\right) \\
\end{bmatrix}
\begin{bmatrix}
x_1(0)
x_2(0)
\end{bmatrix}
$$
If this is invertible then there are two solutions and the null space only contains the zero vector.
$$
\begin{vmatrix}
\left(\frac{6 e }{7}+\frac{1}{7 e^6}\right) & \left(\frac{e}{7}-\frac{1}{7 e^6}\right) \\
\left(\frac{6 e }{7}-\frac{6}{7 e^6}\right) & \left(\frac{e}{7}+\frac{6}{7 e^6}\right) \\
\end{vmatrix}
=
e^{-5}
$$
So the null space only contains the zero vectors.
\newline
The stable subspace of the system occurs when $x(0) = \begin{bmatrix} 0 \\ 0 \end{bmatrix}$

\newpage
\section*{Problem 4}
Find the realizations in controller and observability forms of the transfer function
$$ H(s) = \frac{2s^3+13s^2+31s+32}{s^3+6s^2+11s+6} $$
Give both block diagrams and state-space equations.
\newline
\newline
The controllable form is given by the state equations
$$
A =
\begin{bmatrix}
 0 &  1  &  0 \\
 0 &  0  &  1 \\
-6 & -11 & -6 \\
\end{bmatrix}
\quad
B =
\begin{bmatrix} 0 \\ 0 \\ 1 \end{bmatrix}
\quad
C =
\begin{bmatrix}
32 - 12 & 31 - 22 & 13 - 12
\end{bmatrix}
\quad
D =
\begin{bmatrix}
2
\end{bmatrix}
$$
Reducing we get the following state space equation
$$
A =
\begin{bmatrix}
 0 &  1  &  0 \\
 0 &  0  &  1 \\
-6 & -11 & -6 \\
\end{bmatrix}
\quad
B =
\begin{bmatrix} 0 \\ 0 \\ 1 \end{bmatrix}
\quad
C =
\begin{bmatrix}
20 & 9 & 1
\end{bmatrix}
\quad
D =
\begin{bmatrix}
2
\end{bmatrix}
$$
The block diagram is given by
\begin{figure}[!htb]
\centering
\tikzset{
    block/.style = {draw, fill=white, rectangle, minimum height=1em, minimum width=1em},
    tmp/.style  = {coordinate, node distance=1cm},
    sum/.style= {draw, fill=white, circle, node distance=1cm},
    input/.style = {coordinate, node distance=1cm},
    output/.style= {coordinate, node distance=1cm},
    pinstyle/.style = {pin edge={to-,thin,black}},
}
\begin{tikzpicture}[auto, node distance=2cm,>=latex']
\node [input, name=input](input){};
\node [sum, right of=input](inputSum){};
\node [block, right of=inputSum](integrator1) {$\frac{1}{s}$};
\node [block, right of=integrator1](integrator2) {$\frac{1}{s}$};
\node [block, right of=integrator2](integrator3) {$\frac{1}{s}$};
\node [block, right of=integrator3](b0) {$32$};
\node [sum, right of=b0](finalSum) {};
\node [sum, above of=finalSum](b2pb1) {};
\node [block, left of=b2pb1](b1) {$31$};
\node [tmp, above of=b2pb1](aboveb2pb1) {};
\node [block, left of=aboveb2pb1](b2) {$13$};
\node [sum, below of=inputSum](a2pa1) {};
\node [block, right of=a2pa1](a2) {$6$};
\node [sum, below of=a2pa1](a1pa0) {};
\node [block, right of=a1pa0](a1) {$11$};
\node [tmp, below of=a1pa0](belowa1pa0) {};
\node [block, right of=belowa1pa0](a0) {$6$};
\node [block, right of=finalSum](b3) {$2$};
\node [output, right of=b3](output) {};

\draw [->] (input) -- node{$U(s)$} (inputSum) {};
\draw [->] (inputSum) -- (integrator1) {};
\draw [->] (a2pa1) --node{$-$} (inputSum) {};
\draw [->] (a1pa0) -- (a2pa1) {};
\draw [->] (a0) -| (a1pa0) {};
\draw [->] (a1) -- (a1pa0) {};
\draw [->] (a2) -- (a2pa1) {};

\draw [->] (integrator1) --node[tmp, name=betweenInt1Int2]{} (integrator2) {};
\draw [->] (betweenInt1Int2) |- (a2) {};
\draw [->] (betweenInt1Int2) |- (b2) {};

\draw [->] (integrator2) --node[tmp, name=betweenInt2Int3]{} (integrator3) {};
\draw [->] (betweenInt2Int3) |- (a1) {};
\draw [->] (betweenInt2Int3) |- (b1) {};

\draw [->] (integrator3) --node[tmp, name=betweenInt3B0]{} (b0) {};
\draw [->] (betweenInt3B0) |- (a0) {};
\draw [->] (betweenInt3B0) |- (b0) {};

\draw [->] (b2) -| (b2pb1) {};
\draw [->] (b1) -- (b2pb1) {};

\draw [->] (b2pb1) -- (finalSum) {};
\draw [->] (b0) -- (finalSum) {};

\draw [->] (finalSum) -- (b3) {};
\draw [->] (b3) --node[name=outputLabel]{$Y(s)$} (output) {};
\end{tikzpicture}
\caption*{Controllable Canonical form}
\end{figure}
\newline
The observable form is given by the state equations
$$
A =
\begin{bmatrix}
-6  & 1 & 0 \\
-11 & 0 & 1 \\
-6  & 0 & 0 \\
\end{bmatrix}
\quad
B =
\begin{bmatrix}
13 - 12 \\ 31 - 22 \\ 32 - 12
\end{bmatrix}
\quad
C =
\begin{bmatrix} 1 & 0 & 0 \end{bmatrix}
\quad
D =
\begin{bmatrix}
2
\end{bmatrix}
$$
Reducing we get the following state space equation
$$
A =
\begin{bmatrix}
-6  & 1 & 0 \\
-11 & 0 & 1 \\
-6  & 0 & 0 \\
\end{bmatrix}
\quad
B =
\begin{bmatrix} 1 \\ 9 \\ 20 \end{bmatrix}
\quad
C =
\begin{bmatrix}
1 & 0 & 0
\end{bmatrix}
\quad
D =
\begin{bmatrix}
2
\end{bmatrix}
$$
The block diagram is given by
\begin{figure}[!htb]
\centering
\tikzset{
    block/.style = {draw, fill=white, rectangle, minimum height=1em, minimum width=1em},
    tmp/.style  = {coordinate, node distance=1cm},
    sum/.style= {draw, fill=white, circle, node distance=1cm},
    input/.style = {coordinate, node distance=1cm},
    output/.style= {coordinate, node distance=1cm},
    pinstyle/.style = {pin edge={to-,thin,black}},
}
\begin{tikzpicture}[auto, node distance=2cm,>=latex']
\node [input, name=input, label=left:{$U(s)$}](input) {};
\node [tmp, right of=input](betweenInputB0){};
\node [block, right of=betweenInputB0](b0){$32$};
\node [sum, right of=b0, label=south west:{$-$}](inputSum){};
\node [block, right of=inputSum](integrator1) {$\frac{1}{s}$};
\node [tmp, below of=integrator1](belowI1){};
\node [tmp, below of=belowI1](belowI1x2){};
\node [block, below of=belowI1x2](a0){$6$};
\node [sum, right of=integrator1, label=south west:{$-$}](middleSum){};
\node [block, right of=middleSum](integrator2) {$\frac{1}{s}$};
\node [sum, right of=integrator2, label=south west:{$-$}](rightSum){};
\node [block, right of=rightSum](integrator3) {$\frac{1}{s}$};
\node [block, right of=integrator3](b3) {$2$};
\node [output, right of=b3](output) {};
\node [block, below of=integrator3](a2) {$6$};
\node [tmp, below of=integrator2](belowI2){};
\node [block, below of=belowI2](a1) {$11$};
\node [tmp, above of=integrator2](aboveI2) {};
\node [block, above of=aboveI2](b2) {$13$};
\node [block, above of=integrator1](b1) {$31$};


\draw [->] (input) -- (b0) {};
\draw [->] (betweenInputB0) |- (b1) {};
\draw [->] (betweenInputB0) |- (b2) {};
\draw [->] (b0) -- (inputSum) {};
\draw [->] (inputSum) -- (integrator1) {};
\draw [->] (integrator1) -- (middleSum) {};
\draw [->] (middleSum) -- (integrator2) {};
\draw [->] (integrator2) -- (rightSum) {};
\draw [->] (rightSum) -- (integrator3) {};
\draw [->] (integrator3) --node[name=outputI3]{} (b3) {};
\draw [->] (b3) --node[name=outputLabel]{$Y(s)$} (output) {};

\draw [->] (outputI3) |- (a2) {};
\draw [->] (a2) -| (rightSum) {};
\draw [->] (b2) -| (rightSum) {};

\draw [->] (outputI3) |- (a1) {};
\draw [->] (a1) -| (middleSum) {};
\draw [->] (b1) -| (middleSum) {};

\draw [->] (outputI3) |- (a0) {};
\draw [->] (a0) -| (inputSum) {};


\end{tikzpicture}
\caption*{Observable Canonical form}
\end{figure}
\newline


\end{document}
