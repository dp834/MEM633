\documentclass{article}

\usepackage[margin=.75in]{geometry}
\usepackage{amsmath}
\usepackage{amssymb}
\usepackage[shortlabels]{enumitem}
\usepackage{float}
\usepackage{verbatim}
\usepackage{caption}
\usepackage{subcaption}
\usepackage{tikz}
\usetikzlibrary{shapes,arrows,positioning,calc}

\usepackage{mathtools}
\DeclarePairedDelimiter\norm{\lVert}{\rVert}%

\author{Anthony Siddique, Damien Prieur, Jachin Philip, Obinna Ekeh}
\title{Homework 4\\ MEM 633 \\ Group 1}
\date{}

\begin{document}

\maketitle

\section*{Problem 1}
Consider the system described by the following state equaiton,
$$
\dot{x}(t) = Ax(t) + Bu(t) =
\begin{bmatrix}
-18 & -19 & -15 \\
20 & 21 & 16 \\
-5 & -5 & -4 \\
\end{bmatrix}
x(t)
+
\begin{bmatrix}
-3 \\
 5 \\ 
-2 \\
\end{bmatrix}
u(t)
$$

\begin{enumerate}[(a)]
\item Use the PBH Test to check if the system is controllable.
\newline
\newline
A system is controllable if the $rank\left(\begin{bmatrix} sI - A & b \end{bmatrix}\right) = n \quad \forall s \in \mathcal{C}$
$$
\begin{bmatrix}
s + 18 & 19   &  15 & -3 \\
   -20 & s-21 & -16 &  5 \\
     5 & 5    & s+4 & -2 \\
\end{bmatrix}
$$
Since we only need to check $s$ equal to the eigenvalues of $A$ since $A$ is nonsingular.
$$
\begin{vmatrix}
\lambda + 18 & 19   &  15 \\
   -20 & \lambda-21 & -16 \\
     5 & 5    & \lambda+4 \\
\end{vmatrix}
=
-\lambda^3 - \lambda^2 +5 \lambda -3
$$
$$
\lambda_1 = -3
\quad
\lambda_2 =  1
\quad 
\lambda_3 =  1
$$

Plugging in $s=\lambda_i$ for the PBH test we get
$$
rank(s=-3) =
rank\left(
\begin{bmatrix}
 15 &  19 &  15 & -3 \\
-20 & -24 & -16 &  5 \\
  5 &   5 &   1 & -2 \\
\end{bmatrix}
\right) = 2
$$
$$
rank(s=1) =
rank\left(
\begin{bmatrix}
 19 &  19 &  15 & -3 \\
-20 & -20 & -16 &  5 \\
  5 &   5 &   5 & -2 \\
\end{bmatrix}
\right) = 3
$$
The system is uncontrollable, and the only uncontrollable eigenvalue is $\lambda = -3$.


\item Characterize the controllable subpace using the controllability decomposition approach.
\newline
\newline
TODO


\item Is the system stabilizable? Explain.
\newline
\newline
Yes, since the only uncontrollable eigenvalue is already stable since it's in the left half plane.

\item Design a state feedback controller using the controllability decomposition approach so that the closed-loop system is internally stable.
\newline
\newline

\item Assume the initial state of the system is $x(0) = \begin{bmatrix} 1 & 3 & 2\end{bmatrix}^T$, plot the state response $x(t)$ of the closed-loop system.
\newline
\newline

\end{enumerate}

\section*{Problem 2}
Consider the same system described by the state equation shown in problem 1.
$$
\dot{x}(t) = Ax(t) + Bu(t) =
\begin{bmatrix}
-18 & -19 & -15 \\
20 & 21 & 16 \\
-5 & -5 & -4 \\
\end{bmatrix}
x(t)
+
\begin{bmatrix}
-3 \\
 5 \\ 
-2 \\
\end{bmatrix}
u(t)
$$

\begin{enumerate}[(a)]
\item Find a similarity transformation to transform the state equation to one with a diagonal $A$ matrix.
\newline
\newline

\item Characterize the controllable subspace using the eigenvectors obtained in problem 2(a).
\newline
\newline

\item Check the stabilizability of the system based on the result of problem 2(b).
\newline
\newline

\item Design a state feedback controller using the eigen structure obtained in problem 2(b) so that the closed-loop system is internally stable.
\newline
\newline

\item Assume the initial state of the system is $x(0) = \begin{bmatrix} 1 & 3 & 2\end{bmatrix}^T$, plot the state response $x(t)$ of the closed-loop system.
\newline
\newline

\end{enumerate}


\section*{Problem 3}
\begin{enumerate}[(a)]
\item If $\{A, b, c, d\}, d \neq 0$, is a realization with $H(s) = c(sI-A)^{-1}b +d$, show that
$$ \left\{ A - \frac{bc}{d}, \frac{b}{d}, \frac{-c}{d}, \frac{1}{d} \right\} $$
is a realization for a system with a transfer function $\frac{1}{H(s)}$.
\newline
\newline
Let the system $G(s) = \frac{1}{H(s)}$ be described by $\{\hat{A}, \hat{b}, \hat{c}, \hat{d}\}$.
Its transfer function is given by
$$
G(s) = \hat{c}(sI-\hat{A})^{-1}\hat{b}+\hat{d}
$$
If we expand this and show that $G(s) = \frac{1}{H(s)}$ then the statement is true.
$$\frac{-c}{d} (sI - (A - \frac{bc}{d}))\frac{b}{d} + \frac{1}{d}$$

\item If we are given $\{A, b, c, d\}, d \neq 0$, show that the zeros of $c(sI-A)^{-1}b + d$ are the eigenvalues of the matrix $ A - \frac{bc}{d}$.
\newline
\newline

\end{enumerate}

\end{document}
