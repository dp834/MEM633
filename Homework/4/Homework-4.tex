\documentclass{article}

\usepackage[margin=.75in]{geometry}
\usepackage{amsmath}
\usepackage{amssymb}
\usepackage[shortlabels]{enumitem}
\usepackage{float}
\usepackage{verbatim}
\usepackage{caption}
\usepackage{subcaption}
\usepackage{tikz}
\usetikzlibrary{shapes,arrows,positioning,calc}

\usepackage{mathtools}
\DeclarePairedDelimiter\norm{\lVert}{\rVert}%

\author{Anthony Siddique, Damien Prieur, Jachin Philip, Obinna Ekeh}
\title{Homework 4\\ MEM 633 \\ Group 1}
\date{}

\begin{document}

\maketitle

\section*{Problem 1}
Consider the system described by the following state equaiton,
$$
\dot{x}(t) = Ax(t) + Bu(t) =
\begin{bmatrix}
-18 & -19 & -15 \\
20 & 21 & 16 \\
-5 & -5 & -4 \\
\end{bmatrix}
x(t)
+
\begin{bmatrix}
-3 \\
 5 \\
-2 \\
\end{bmatrix}
u(t)
$$

\begin{enumerate}[(a)]
\item Use the PBH Test to check if the system is controllable.
\newline
\newline
A system is controllable if the $rank\left(\begin{bmatrix} sI - A & b \end{bmatrix}\right) = n \quad \forall s \in \mathcal{C}$
$$
\begin{bmatrix}
s + 18 & 19   &  15 & -3 \\
   -20 & s-21 & -16 &  5 \\
     5 & 5    & s+4 & -2 \\
\end{bmatrix}
$$
Since we only need to check $s$ equal to the eigenvalues of $A$ since $A$ is nonsingular.
$$
\begin{vmatrix}
\lambda + 18 & 19   &  15 \\
   -20 & \lambda-21 & -16 \\
     5 & 5    & \lambda+4 \\
\end{vmatrix}
=
-\lambda^3 - \lambda^2 +5 \lambda -3
$$
$$
\lambda_1 = -3
\quad
\lambda_2 =  1
\quad
\lambda_3 =  1
$$

Plugging in $s=\lambda_i$ for the PBH test we get
$$
rank(s=-3) =
rank\left(
\begin{bmatrix}
 15 &  19 &  15 & -3 \\
-20 & -24 & -16 &  5 \\
  5 &   5 &   1 & -2 \\
\end{bmatrix}
\right) = 2
$$
$$
rank(s=1) =
rank\left(
\begin{bmatrix}
 19 &  19 &  15 & -3 \\
-20 & -20 & -16 &  5 \\
  5 &   5 &   5 & -2 \\
\end{bmatrix}
\right) = 3
$$
The system is uncontrollable, and the only uncontrollable eigenvalue is $\lambda = -3$.


\item Characterize the controllable subpace using the controllability decomposition approach.
\newline
\newline
Let $T = \begin{bmatrix} T_1 & T_2 \end{bmatrix}$, where $T_1 = \begin{bmatrix} b & Ab & ... & A^{r-1}b \end{bmatrix}$ and $T_2$ be chosen such that $T$ is full rank.
Since we know that only 2 eigenvalues are controllable we only need the first two columns of $T_1$
$$ T_1 = \begin{bmatrix} b & Ab \end{bmatrix}
=
\begin{bmatrix}
-3 & -11 \\
5 & 13 \\
-2 & -2 \\
\end{bmatrix}
$$
$$ T_2 = \begin{bmatrix} 0 \\ 0 \\ 1 \end{bmatrix} $$
$$
T =
\begin{bmatrix}
-3 & -11 & 0 \\
 5 &  13 & 0 \\
-2 &  -2 & 1 \\
\end{bmatrix}
\implies T^{-1} =
\frac{1}{16}
\begin{bmatrix}
13 & 11 &  0 \\
-5 & -3 &  0 \\
16 & 16 & 16 \\
\end{bmatrix}
$$
Let $\hat{A} = T^{-1}AT, \hat{b} = T^{-1}b$ then we get
$$
\hat{A} =
\begin{bmatrix}
0 & -1 & -\frac{19}{16} \\
1 &  2 &  \frac{27}{16} \\
0 &  0 & -3 \\
\end{bmatrix}
\quad
\hat{b} = \begin{bmatrix} 1 \\ 0 \\ 0 \end{bmatrix}
$$

\item Is the system stabilizable? Explain.
\newline
\newline
Yes, since the only uncontrollable eigenvalue is already stable since it's in the left half plane.

\item Design a state feedback controller using the controllability decomposition approach so that the closed-loop system is internally stable.
\newline
\newline
Looking at the controllable subsystem given by the first $2x2$ subsystem we have
$$
A =
\begin{bmatrix}
0 & -1 \\
1 &  2 \\
\end{bmatrix}
\quad
B =
\begin{bmatrix}
1 \\
0 \\
\end{bmatrix}
$$
The $A$ matrix has eigenvalues $\lambda_1 = 1, \lambda_2 = 1$ as we expect.
To control this system we can use pole placement to move the eigenvalues somewhere in the left half plane.
Let those poles be $\{-5, -7\}$
$$ det(\lambda I - (A-bF)) = (\lambda+5)(\lambda+7) $$
$$
\begin{vmatrix}
\lambda + f_1 & \lambda + 1 + 2f_2 \\
-1 & \lambda - 2 \\
\end{vmatrix}
= (\lambda+5)(\lambda+7)
$$
$$
\lambda^2 + \lambda(f_1 - 2) + 1 -2f_1 +f_2
= \lambda^2 + 12 \lambda +35
$$
$$ f_1 = 14 \quad f_2 = 62 $$
Putting this into the larger system we get $u=-Fx$
$$
\begin{bmatrix}
0 & -1 & -\frac{19}{16} \\
1 &  2 &  \frac{27}{16} \\
0 &  0 & -3 \\
\end{bmatrix}
\hat{x}(t)
-
\begin{bmatrix} 1 \\ 0 \\ 0 \end{bmatrix}
\begin{bmatrix}14 & 62 & 0 \end{bmatrix}
=
\begin{bmatrix}
-14 & -63 & -\frac{19}{16} \\
  1 &   2 &  \frac{27}{16} \\
  0 &   0 & -3 \\
\end{bmatrix}
$$
Which has eigenvalues $-7,-5,-3$ as we expect.
So the internal system is stable.

\newpage
\item Assume the initial state of the system is $x(0) = \begin{bmatrix} 1 & 3 & 2\end{bmatrix}^T$, plot the state response $x(t)$ of the closed-loop system.
\newline
\newline
We can undo the similarity transform with
$$ A = T\hat{A}T^{-1} \quad b = T\hat{b} $$
We can see that the feedback control values are unaffected by the similarity transformation so we get
$$
\dot{x}(t) =
\left(
\begin{bmatrix}
-18 & -19 & -15 \\
20 & 21 & 16 \\
-5 & -5 & -4 \\
\end{bmatrix}
-
\begin{bmatrix}
-3 \\
 5 \\
-2 \\
\end{bmatrix}
\begin{bmatrix} 14 & 62 & 0 \end{bmatrix}
\right)x(t)
=
\begin{bmatrix}
 24 &  167 & -15 \\
-50 & -289 &  16 \\
 23 &  119 &  -4 \\
\end{bmatrix}
x(t)
$$
Which has a solution
$$x(t) = e^{A t}x(0)$$
\begin{figure}[!ht]
\centering
\includegraphics[scale=.75]{{images/p1_closed_loop_response}.png}
\end{figure}
\end{enumerate}

\newpage
\section*{Problem 2}
Consider the same system described by the state equation shown in problem 1.
$$
\dot{x}(t) = Ax(t) + Bu(t) =
\begin{bmatrix}
-18 & -19 & -15 \\
20 & 21 & 16 \\
-5 & -5 & -4 \\
\end{bmatrix}
x(t)
+
\begin{bmatrix}
-3 \\
 5 \\
-2 \\
\end{bmatrix}
u(t)
$$

\begin{enumerate}[(a)]
\item Find a similarity transformation to transform the state equation to one with a diagonal $A$ matrix.
\newline
\newline
We know that the eigenvalues of this system are
$$ \lambda_1 = -3 \quad \lambda_2 = 1 \quad \lambda_3 = 1 $$
Since we have a repeated eigenvalue we will need to find a generalized eigenvector for the third eigenvector.
So we are really finding the Jordan decomposition.
Our normal eigenvalues are
$$
\lambda = -3 \to
\begin{bmatrix}
14 \\
-15 \\
5 \\
\end{bmatrix}
\quad
\lambda = 1 \to
\begin{bmatrix}
1 \\
-1 \\
0 \\
\end{bmatrix}
$$
To find the generalized eigenvalue for the repeated eigenvalue we need to solve
$$ (A-\lambda I)v_3 = v_2 $$
Where $v_3$ is the generalized eigenvalue and $v_2$ is the standard eigenvector for the associated eigenvalue.
Solving this we get
$$
\lambda = 1 \to
\begin{bmatrix}
\frac{1}{4} \\
0 \\
-\frac{1}{4} \\
\end{bmatrix}
$$
So we have our generalized eigenvectors as
$$
S =
\begin{bmatrix}
14 & -1 & \frac{1}{4} \\
-15 & 1 & 0 \\
5 & 0 & -\frac{1}{4} \\
\end{bmatrix}
$$
The diagonal matrix will be block diagonal since we have a repeated eigenvalue
$$
J =
\begin{bmatrix}
-3 & 0 & 0 \\
 0 & 1 & 1 \\
 0 & 0 & 1 \\
\end{bmatrix}
$$
So we have
$$ A = SJS^{-1} $$
So our similarity transform is
$$ T = S^{-1} \implies T^{-1} = S $$


\item Characterize the controllable subspace using the eigenvectors obtained in problem 2(a).
\newline
\newline
An eigenvector is uncontrollable if and only if $qA = \lambda q$ and $qb = 0$
We can check them all at once by looking at $TA$ and $Tb$
$$
T =
\begin{bmatrix}
\frac{1}{4} & \frac{1}{4} & \frac{1}{4} \\
\frac{15}{4} & \frac{19}{4} & \frac{15}{4} \\
 5 & 5 & 1 \\
\end{bmatrix}
\quad
TA =
\begin{bmatrix}
-\frac{3}{4} & -\frac{3}{4} & -\frac{3}{4} \\
\frac{35}{4} & \frac{39}{4} & \frac{19}{4} \\
 5 & 5 & 1 \\
\end{bmatrix}
\quad
Tb =
\begin{bmatrix}
0 \\
5 \\
8 \\
\end{bmatrix}
$$
We can see that the first row is uncontrollable and it corresponds to the eigenvalue $\lambda = -3$
Looking at our system in the diagonal case we taking just the controllable eigenvalues
$$
\hat{A} =
\begin{bmatrix}
1 & 1 \\
0 & 1 \\
\end{bmatrix}
\quad
\hat{B} =
\begin{bmatrix}
5 \\
8 \\
\end{bmatrix}
$$
This is our controllable subspace in the transformed domain.


\item Check the stabilizability of the system based on the result of problem 2(b).
\newline
\newline
Since the only uncontrollable eigenvector is already stable since $\lambda = -3$ we can stabilize the system as the other two eigenvectors are controllable.

\item Design a state feedback controller using the eigen structure obtained in problem 2(b) so that the closed-loop system is internally stable.
\newline
\newline
Using our controllable subsystem we have
$$
\begin{bmatrix}
1 & 1 \\
0 & 1 \\
\end{bmatrix}
x(t) -
\begin{bmatrix}
5 \\
8 \\
\end{bmatrix}
\begin{bmatrix} f1 & f2 \end{bmatrix}
x(t)
$$
Using pole placement with the same poles as before, $\{-5, -7\}$.
$$ det(\lambda I - (A-bF)) = (\lambda+5)(\lambda+7) $$
This gives us
$$ F = \begin{bmatrix} 6 & -2 \end{bmatrix} $$
With a closed loop system of
$$
\begin{bmatrix}
-29 & 11 \\
-48 & 17
\end{bmatrix}
$$
Which has eigenvalues $\lambda = -5$ and $\lambda = -7$ as we expect.
Putting this back into the full system we have
$$ \hat{A} - \hat{b}\begin{bmatrix} 0 \\ 6 \\ -2 \end{bmatrix} $$
And we know that converting back to our original domain doesn't effect the control.
Our closed controller gives us
$$
\dot{x}(t) =
\begin{bmatrix}
-18 & -1 & -21 \\
 20 & -9 &  26 \\
 -5 &  7 &  -8 \\
\end{bmatrix}
x(t)
$$
This has different eigenvalues which is due to the fact that we didn't have a fully diagonal matrix due to the repeated root requiring a jordan block.
The final eigenvalues of our system are
$$\lambda = -3 \quad \lambda = -16 - \sqrt{241} \quad \lambda = -16 + \sqrt{241}$$

\item Assume the initial state of the system is $x(0) = \begin{bmatrix} 1 & 3 & 2\end{bmatrix}^T$, plot the state response $x(t)$ of the closed-loop system.
\newline
\newline
Again the system has the solution
$$x(t) = e^{A t}x(0)$$
\begin{figure}[!ht]
\centering
\includegraphics[scale=.5]{{images/p2_closed_loop_response}.png}
\end{figure}

\end{enumerate}


\newpage
\section*{Problem 3}
\begin{enumerate}[(a)]
\item If $\{A, b, c, d\}, d \neq 0$, is a realization with $H(s) = c(sI-A)^{-1}b +d$, show that
$$ \left\{ A - \frac{bc}{d}, \frac{b}{d}, \frac{-c}{d}, \frac{1}{d} \right\} $$
is a realization for a system with a transfer function $\frac{1}{H(s)}$.
\newline
\newline
Let the system $G(s) = \frac{1}{H(s)}$ be described by $\{\hat{A}, \hat{b}, \hat{c}, \hat{d}\}$.
Its transfer function is given by
$$
G(s) = \hat{c}(sI-\hat{A})^{-1}\hat{b}+\hat{d}
$$
If we expand this and show that $G(s) = \frac{1}{H(s)}$ then the statement is true.
$$\frac{-c}{d} (sI - (A - \frac{bc}{d}))\frac{b}{d} + \frac{1}{d}$$
% TODO actually do this...

\item If we are given $\{A, b, c, d\}, d \neq 0$, show that the zeros of $c(sI-A)^{-1}b + d$ are the eigenvalues of the matrix $ A - \frac{bc}{d}$.
\newline
\newline
The eigenvalues of $A-\frac{bc}{d}$ can be expanded to
$$
A - \frac{1}{d}
\begin{bmatrix}
b_1 c_1 & b_1 c_2 & ... & b_1 c_n \\
b_2 c_1 & b_2 c_2 & ... & b_2 c_n \\
...  & ... & ... & ... \\
b_n c_1 & b_n c_2 & ... & b_n c_n \\
\end{bmatrix}
$$
Let $q_{i,j} = \frac{b_ic_j}{d}$ and $Q$ be the matrix of $q_{i,j}$.
$$ |\lambda I - (A-Q)| = 0 $$
A matrix where
$$
J(i,j) =
\begin{cases}
\lambda  + q_{i,i} - a_{i,i} & i = j \\
q_{i,j} - a_{i,j} & i \neq j \\
\end{cases}
$$
Which we can find the determinant of if we take the cofactor expansion of each element.
% TODO reduce this down, or show using an argument about the state space vs laplace space meanings of eigenvalues/zeros
\newline
\newline
$$c \frac{cofactor(sI-A)^T}{det(sI-A)} b + d $$
Let $\mathcal{A}$ be the adjoint of the matrix $(sI-A)$ and $\mathcal{C}_{i,j}$ be the cofactor of the element $\{i,j\}$
$$c \frac{\mathcal{A}}{det(sI-A)} b + d $$
Since we only care about zeros we can remove the denominator
$$c \mathcal{A} b = - d $$
Performing the matrix multiplication we get
$$
c_1 (b_1 \mathcal{C}_{1,1} + b_2\mathcal{C}_{2,1} + ... + b_n\mathcal{C}_{n,1}) +
c_2 (b_1\mathcal{C}_{1,2} + b_2\mathcal{C}_{2,2} + ... + b_n\mathcal{C}_{n,2}) +
...
c_n (b_1\mathcal{C}_{1,n} + b_2\mathcal{C}_{2,n} + ... + b_n\mathcal{C}_{n,n})
=
-d
$$
Which can be compact written as
$$\sum_j^n c_j(\sum_i^n b_i \mathcal{C}_{i,j}) = -d$$
Dividing both sides by $d$ and the same definition for $q$ we can see that we get
$$\sum_j^n \sum_i^n q_{i,j} \mathcal{C}_{i,j} = -1 $$
This is the same sum we are searching for when we are solving for the determinant and therefore the solutions will be the same.
\newline
Q.E.D



\end{enumerate}

\end{document}
