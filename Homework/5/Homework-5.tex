\documentclass{article}

\usepackage[margin=.75in]{geometry}
\usepackage{amsmath}
\usepackage{amssymb}
\usepackage[shortlabels]{enumitem}
\usepackage{float}
\usepackage{verbatim}
\usepackage{caption}
\usepackage{subcaption}
\usepackage{tikz}
\usetikzlibrary{shapes,arrows,positioning,calc}

\usepackage{mathtools}
\DeclarePairedDelimiter\norm{\lVert}{\rVert}%

\author{Anthony Siddique, Damien Prieur, Jachin Philip, Obinna Ekeh}
\title{Homework 5\\ MEM 633 \\ Group 1}
\date{}

\begin{document}

\maketitle

\section*{Problem 1}
Find a minimal controller-form (NOT a block controller form) realization of
\[
H(s) =
\setlength{\delimitershortfall}{0pt}
\begin{bmatrix}
\frac{1}{(s-1)^2} & \frac{1}{(s-1)(s+3)} \\[0.5em]
\frac{-6}{(s-1)(s+3)^2} & \frac{s-2}{(s+3)^2} \\[0.5em]
\end{bmatrix}
\]
\newline
\newline
First we must find a Matrix Fraction Description (MFD) of \(H(s) = N(s)D(s)^{1}\)
Using the Smith-McMillan form we start by writing \(H(s) = \frac{N(s)}{d(s)}\) where \(d(s)\) is the monic least common multiple of the denominators of the entries of \(H(s)\).
\[
d(s) = (s-1)^2(s+3)^2 \implies
N(s) =
\begin{bmatrix}
(s+3)^2 & (s-1)(s+3) \\
-6(s-1) & (s-1)^2(s-2) \\
\end{bmatrix}
\]
Now we need to find the Smith form of \(H(s)\) by elementary row and column operations.
First get the first entry to be a constant then we can easily remove all off diagonal entries.
\[
\begin{bmatrix}
1 & \frac{s+7}{6} \\
0 & 1 \\
\end{bmatrix}
\begin{bmatrix}
(s+3)^2 & (s-1)(s+3) \\
-6(s-1) & (s-1)^2(s-2) \\
\end{bmatrix}
\begin{bmatrix}
1 & \frac{s^2-3s}{6}\\
0 & 1 \\
\end{bmatrix}
=
\begin{bmatrix}
16 & \frac{1}{6}(s^4 + 3s^3 -s^2 -3s - 32) \\
-6(s-1) & 2(s-1) \\
\end{bmatrix}
\]
We can rescale to make it a bit easier to see
\[
\begin{bmatrix}
6 & 0\\
0 & 1\\
\end{bmatrix}
\begin{bmatrix}
16 & \frac{1}{6}(s^4 + 3s^3 -s^2 -3s - 32) \\
-6(s-1) & 2(s-1) \\
\end{bmatrix}
\begin{bmatrix}
\frac{1}{6} & \frac{1}{3} \\
0 & 1 \\
\end{bmatrix}
=
\begin{bmatrix}
16 & s^4 + 3s^3 -s^2 -3s \\
(s-1) & 0 \\
\end{bmatrix}
\]
Now we can remove the off diagonal entries with
$$
\begin{bmatrix}
1 & 0 \\
s-1 & 16\\
\end{bmatrix}
\begin{bmatrix}
16 & s^4 + 3s^3 -s^2 -3s \\
(s-1) & 0 \\
\end{bmatrix}
\begin{bmatrix}
1 & \frac{-1}{16}(s^4 + 3s^3 -s^2 -3s)  \\
0 & 1 \\
\end{bmatrix}
=
\begin{bmatrix}
16 & 0 \\
0 & s^5+2s^4-4s^3-2s^2+3s \\
\end{bmatrix}
$$
We can combine the elementary operations to find $U$ and $V$
\[
U =
\begin{bmatrix}
6 & s+7\\
7(s-1) & (s+3)^2\\
\end{bmatrix}
\quad
V =
\begin{bmatrix}
\frac{1}{6} & \frac{-1}{96}(s^4+3s^3-17s^2+45s-32) \\
0 & 1 \\
\end{bmatrix}
\]
Which we can verify by checking $U(s)d(s)H(s)V(s) = \Lambda(s)$.
We can get $U_1$ and $U_2$ by noting that $H(s) = U_1(s)\frac{\Lambda(s)}{d(s)}U_2(s)$
So $U_1(s) = U^{-1}$ and $U_2 = V^{-1}$
$$
U_1(s) =
\frac{1}{16}
\begin{bmatrix}
\frac{1}{6}(s+3)^2 & \frac{-1}{6}(s+7) \\
s-1 & 1 \\
\end{bmatrix}
\quad
U_2(s) =
\begin{bmatrix}
6 & \frac{1}{16}(s^4+3s^3-17s^2+45s-32) \\
0 & 1 \\
\end{bmatrix}
$$
We can get the Smith-McMillan Form of $H(s)$ with $H(s) = U_1(s) M(s) U_2(s)$
Which we can easily get our $\epsilon$'s and $\psi$'s from
$$
M(s) =
\begin{bmatrix}
\frac{16}{(s-1)^2(s+3)^2} & 0 \\
0 & \frac{s(s+1)}{s+3} \\
\end{bmatrix}
\implies
\epsilon_1 = 16, \quad \epsilon_2 = s^2+6s+15 \quad
\psi_1 = (s-1)^2(s+3)^2, \quad \psi_2 = (s+3)
$$
With the $\psi$'s we can determine that the minimal realization of $H(s)$ is $n_{min}=5$.
Expanding $M(s) = E(s)\Psi_R(s)^{-1}$ we get $H(s) = U_1(s)E(s)[U_2(s)^{-1}\Psi_R(s)]^{-1}$ which gives us our MFD form
$ H(s) = N_0(s)D_0(s)^{-1} $ which will have minimal degree.
$$
H(s) =
\begin{bmatrix}
\frac{1}{6}(s+3)^2 & \frac{-1}{6}(s+7) \\
s-1 & 1 \\
\end{bmatrix}
\begin{bmatrix}
16 & 0 \\
0 & s^2 +6s + 15 \\
\end{bmatrix}
\left(
\begin{bmatrix}
6 & \frac{1}{16}(s^4+3s^3-17s^2+45s-32) \\
0 & 1 \\
\end{bmatrix}
\begin{bmatrix}
(s-1)^2(s+3)^2 & 0 \\
0 & (s+3) \\
\end{bmatrix}
\right)^{-1}
$$
So we now have a MFD of $H(s)$.
$$
H(s) =
\begin{bmatrix}
\frac{1}{6} (s-1) (s+3)^2 & -\frac{1}{96} (s-1) s (s+1) (s+7) \\
-(s-1)^2 & \frac{1}{16} (s-1) s (s+1) \\
\end{bmatrix}
\begin{bmatrix}
\frac{1}{6} (s-1)^3 (s+3)^2 & -\frac{1}{96} (s-1) (s+3) \left(s^4+3 s^3-17 s^2+45 s-32\right) \\
0 & (s-1) (s+3) \\
\end{bmatrix}
^{-1}
$$
With this we can now create the state space model.
We write $D(s) = D_{hc}S(s) + D_{IC}\Psi(s)$ where $S(s)$ is a diagonal matrix of $s^k$ for each column where $k$ is the largest degree of that column and $\Psi(s)$ recreates the remaining terms.
$$
D(s) =
\begin{bmatrix}
\frac{s^5}{6} & \frac{-s^6}{96} \\
  0 &   0 \\
\end{bmatrix}
+
\begin{bmatrix}
\frac{s^4}{2}-s^3-\frac{5 s^2}{3}+\frac{7 s}{2}-\frac{3}{2} & -\frac{5 s^5}{96}+\frac{7 s^4}{48}-\frac{s^3}{48}-\frac{109s^2}{96}+\frac{199 s}{96}-1 \\
0 & s^2+2 s-3 \\
\end{bmatrix}
$$
Not sure why $D_{hc}$ is singluar here, did I mess up on the Smith McMillian process?
\newline
Using the $S(s)$ and $\Psi(s)$ matrices we get
$$
D(s) =
\begin{bmatrix}
\frac{1}{6} & \frac{-1}{96} \\
0 & 0 \\
\end{bmatrix}
\begin{bmatrix}
s^5 & 0 \\
0 & s^6 \\
\end{bmatrix}
+
\setcounter{MaxMatrixCols}{11}
\begin{bmatrix}
\frac{1}{2} & -1 & \frac{-5}{3} & \frac{7}{2} & \frac{-3}{2} & \frac{-5}{96} & \frac{7}{48} & \frac{-1}{48} & \frac{-109}{96} & \frac{199}{96} & -1 \\
          0 &  0 &            0 &           0 &            0 &             0 &            0 &             0 &               1 &              2 & -3 \\
\end{bmatrix}
\begin{bmatrix}
s^4 & 0 \\
s^3 & 0 \\
s^2 & 0 \\
  s & 0 \\
  1 & 0 \\
  0 & s^5 \\
  0 & s^4 \\
  0 & s^3 \\
  0 & s^2 \\
  0 & s \\
  0 & 1 \\
\end{bmatrix}
$$
We can also write the $N(s)$ matrix in terms of $\Psi(s)$
$$
N(s) =
\setcounter{MaxMatrixCols}{11}
\begin{bmatrix}
0 & \frac{1}{6} & \frac{5}{6} & \frac{1}{2} & \frac{-3}{2} & 0 & \frac{-1}{96} & \frac{-7}{96} & \frac{1}{96} & \frac{7}{96} & 0 \\
0 & 0& -1 &2 & -1 & 0 & 0 & \frac{1}{16} & 0 & \frac{-1}{16} & 0 \\
\end{bmatrix}
\begin{bmatrix}
s^4 & 0 \\
s^3 & 0 \\
s^2 & 0 \\
  s & 0 \\
  1 & 0 \\
  0 & s^5 \\
  0 & s^4 \\
  0 & s^3 \\
  0 & s^2 \\
  0 & s \\
  0 & 1 \\
\end{bmatrix}
$$
We can now find the controller form realization $\{A_c,B_c,C_c\}$
First we have $A_c^o$ and $B_c^o$
$$
A_c^o =
\setcounter{MaxMatrixCols}{11}
\begin{bmatrix}
0 & 0 & 0 & 0 & 0 & 0 & 0 & 0 & 0 \\
1 & 0 & 0 & 0 & 0 & 0 & 0 & 0 & 0 \\
0 & 1 & 0 & 0 & 0 & 0 & 0 & 0 & 0 \\
0 & 0 & 1 & 0 & 0 & 0 & 0 & 0 & 0 \\
0 & 0 & 0 & 1 & 0 & 0 & 0 & 0 & 0 \\
0 & 0 & 0 & 0 & 0 & 0 & 0 & 0 & 0 \\
0 & 0 & 0 & 0 & 1 & 0 & 0 & 0 & 0 \\
0 & 0 & 0 & 0 & 0 & 1 & 0 & 0 & 0 \\
0 & 0 & 0 & 0 & 0 & 0 & 1 & 0 & 0 \\
0 & 0 & 0 & 0 & 0 & 0 & 0 & 1 & 0 \\
0 & 0 & 0 & 0 & 0 & 0 & 0 & 0 & 1 \\
\end{bmatrix}
\quad
B_c^o =
\begin{bmatrix}
1 & 0 \\
0 & 0 \\
0 & 0 \\
0 & 0 \\
0 & 0 \\
0 & 1 \\
0 & 0 \\
0 & 0 \\
0 & 0 \\
0 & 0 \\
0 & 0 \\
\end{bmatrix}
$$
With this and the relationships
$$ A_c = A_c^o - B_c^o D_{hc}^{-1}D_{Ic} \quad B_c = B_c^oD_{hc}^{-1} \quad C_c = N_{ic} $$
Due the the inverse of $D_{hc}$ not existing since the one I have is singular It can't be computed.
We can find the state space representation
$$
A_c =
\begin{bmatrix}
9
\end{bmatrix}
$$

\newpage
\section*{Problem 2}
Let
\[
A_c =
\begin{bmatrix}
-a_1 & -a_2 & -a_3 \\
   1 &    0 &    0 \\
   0 &    1 &    0 \\
\end{bmatrix}
\]
Show that the unimodular matrices
\[
U(s) =
\begin{bmatrix}
0 & 0 & 1 \\
0 & 1 & s \\
1 & s+a_1 & s^2+a_1s+a_2 \\
\end{bmatrix}
, \quad
V(s) =
\begin{bmatrix}
0 & -1 & s^2 \\
-1 & 0 & s \\
0 & 0 & 1 \\
\end{bmatrix}
\]
will convert \(sI-A_c\) to its Smith form.
Generalize the above result to its matrix case;
i.e., each element in \(A_c\) is replaced by a corresponding \(n\)x\(n\) matix.
\newline
\newline
We can check the first case by just computing $U(s)A_cV(s)$
$$
U(s)A_cV(s) =
\begin{bmatrix}
1 & 0 & 0 \\
0 & 1 & 0 \\
0 & 0 & s^3 + a_1 s^2 + a_2 s + a_3 \\
\end{bmatrix}
$$
Which we can see is clearly in Simth form.
If we rewrite the matricies with matrix indexes for $a_i$ and note that $I$ is the identity matrix that is $n$x$n$
$$
A_c =
\begin{bmatrix}
-A_1 & -A_2 & -A_3 \\
   I &    0 &    0 \\
   0 &    I &    0 \\
\end{bmatrix}
\quad
U(s) =
\begin{bmatrix}
0 & 0 & I \\
0 & I & sI \\
I & sI+A_1 & s^2I+sA_1 + A_2 \\
\end{bmatrix}
\quad
V(s) =
\begin{bmatrix}
0 & -I & s^2I \\
-I & 0 & sI \\
0 & 0 & I \\
\end{bmatrix}
$$
Computing this we get
$$
\begin{bmatrix}
0 & 0 & I \\
0 & I & sI \\
I & sI+A_1 & s^2I+sA_1 + A_2 \\
\end{bmatrix}
\begin{bmatrix}
sI+A_1 & A_2 & A_3 \\
  -I &   sI &    0 \\
   0 &   -I &   sI \\
\end{bmatrix}
\begin{bmatrix}
0 & -I & s^2I \\
-I & 0 & sI \\
0 & 0 & I \\
\end{bmatrix}
$$
Performing the right multiplication first we get
$$
\begin{bmatrix}
0 & 0 & I \\
0 & I & sI \\
I & sI+A_1 & s^2I+sA_1 + A_2 \\
\end{bmatrix}
\begin{bmatrix}
-A_2 & sI+A_1 & (sI+A_1) s^2 +A_2s + A_3 \\
-sI &      I & 0 \\
  I &      0 & 0 \\
\end{bmatrix}
$$
Then we get
$$
\begin{bmatrix}
I & 0 & 0 \\
0 & I & 0 \\
-A_2 -Is^2-A_1s + s^2I +sA_1+A_2 & sI+A_1 - sI - A_1 & (sI+A_1) s^2 +A_2s + A_3 \\
\end{bmatrix}
$$
Which cancels as we expect to
$$
\begin{bmatrix}
I & 0 & 0 \\
0 & I & 0 \\
0 & 0 & s^3I+ A_1s^2 + A_2s + A_3 \\
\end{bmatrix}
$$
Which is again in Smith form.
This shows that the generalized matricies work for the matrix case of $A$.


\end{document}
